\documentclass[a4paper, slovene]{article}
\usepackage[slovene]{babel}
\usepackage[style=numeric, backend=biber]{biblatex}
\usepackage{amsfonts}
\usepackage{hyperref}

\title{Obrnljive funkcije so sfere v svetu}
\author{
  Timon Najdovski \and
  Mentor: prof.~dr.~Andrej~Bauer \and
  Somentor: asist.~-~razisk.~PhD.~Egbert~Marteen~Rijke
}


\date{\today}
\addbibresource{../literatura.bib}
\usepackage{amsmath}
\usepackage{comment}
\usepackage{quiver}
\usepackage{hyperref}
\usepackage{mathtools}

\newenvironment{konstrukcija}{\paragraph{Konstrukcija:}}{\hfill$\square$}

\newcommand{\N}{\mathbb N}
\newcommand{\Z}{\mathbb Z}

\newcommand{\typejudgment}[2]{#1 : #2}
\newcommand{\defun}[3]{#1 \,\colon #2 \to #3}
\newcommand{\depfun}[3]{#1 \,\colon #2 \to_{\ast} #3}
\newcommand{\eqtype}[3]{#1 \,\colon #2 = #3}
\newcommand{\homotopy}[3]{#1 \,\colon #2 \sim #3}
\newcommand{\dequiv}[3]{#1 \,\colon #2 \simeq #3}
%\newcommand{\sumtype}[3]{\sum\left( \typejudgment{#1}{#2} \right) #3}
%\newcommand{\pitype}[3]{\prod \left( \typejudgment{#1}{#2} \right) #3}
\newcommand{\sumtype}[3]{\textstyle \sum_{\left( \typejudgment{#1}{#2} \right)} #3}
\newcommand{\pitype}[3]{\textstyle \prod_{\left( \typejudgment{#1}{#2} \right)} #3}
\newcommand{\typename}[1]{\textnormal{\textsf{#1}}}
\newcommand{\emptytype}{\textnormal{\textbf{0}}}
\newcommand{\unittype}{\textnormal{\textbf{1}}}
\newcommand{\type}[1]{#1\ \typename{type}}
\newcommand{\proj}[1]{\typename{pr}_{#1}}
\newcommand{\natzero}{\typename{0}}
\newcommand{\natsucc}{\typename{S}}
\newcommand{\id}[1]{\typename{id}_{#1}}
\newcommand{\Id}[1]{\typename{Id}_{#1}}
\newcommand{\ap}[1]{\typename{ap}_{#1}\,}
\newcommand{\htpyarr}[1]{\xrightarrow[\sim]{#1}}

%%%Common types
\newcommand{\isinv}{\typename{inverse}}
\newcommand{\iscohinv}{\typename{is-coh-invertible}}
\newcommand{\isequiv}{\typename{is-equiv}}
\newcommand{\iscontr}{\typename{is-contr}}
\newcommand{\isprop}{\typename{is-prop}}
\newcommand{\refl}[1]{\typename{refl}_{#1}}
\newcommand{\map}{\typename{map}}
\newcommand{\sect}{\typename{sec}}
\newcommand{\ret}{\typename{ret}}
\newcommand{\tot}{\typename{tot}}
\newcommand{\transp}[1]{\typename{tr}_{#1}}
\newcommand{\fib}[2]{\typename{fib}_{#1}\left(#2\right)}
\newcommand{\sphere}[1]{\mathbb{S}^{#1}}
\newcommand{\krog}{\sphere{1}}
\newcommand{\base}{\typename{base}}
\newcommand{\lop}{\typename{loop}}
\newcommand{\surf}{\typename{surf}}
\newcommand{\ind}[1]{\typename{ind}_{#1}}
\newcommand{\invmap}{\typename{inv-map}}

%%% Local Variables:
%%% mode: latex
%%% TeX-master: "./Najdovski-27191110-2023"
%%% End:


\begin{document}
\maketitle

\begin{section}{Osnove Martin-Löfove odvisne teorije tipov}
  Sledeč \cite{rijke2022introduction} bomo v tem poglavju predstavili
  osnovne koncepte Martin-Löfove odvisne teorije tipov. Zaradi njihove razsežnosti se ne
  bomo spustili v podrobnosti formalnih definicij, vendar bomo osnove opisali v nekoliko
  neformalnem slogu.

  Osnovni objekti, s katerimi bomo delali, so \emph{tipi}, ki jih bomo označevali z
  velikimi tiskanimi črkami, kot so \(A, B\) in \(C\). Tipi lahko vsebujejo
  \emph{elemente}, kar bomo označili z zapisom \(\typejudgment{x}{A}\), ki ga preberemo
  kot \emph{x je tipa A}. Zapis \(\typejudgment{x}{A}\) je v mnogih pogledih
  podoben relaciji \(x \in A\) iz formalizma teorije množic, vendar se od nje razlikuje
  v nekaj pomembnih pogledih. Elementi v teoriji tipov obstajajo samo znotraj svojih
  tipov, kar pomeni, da za element \(\typejudgment{x}{A}\) vprašanje
  ``Ali velja \(\typejudgment{x}{B}\)?'' sploh ni smiselno. V nasprotju s tem
  lahko v teoriji tipov \(\typejudgment{x}{A}\) le predpostavimo, ali pa
  konstruiramo po nekem zaporedju sklepov.

  Iz znanih tipov lahko tvorimo mnoge sestavljene tipe, nekaj od katerih naštejemo spodaj.
  Sestavljene tipe podamo s sklepi, ki nam omogočajo delo z njihovimi elementi, in
  z enačbami, ki nam omogočajo, da en izraz nadomestimo z drugim.

  \begin{itemize}
  \item \emph{Koprodukt} \(A + B\), za katerega veljata sklepa:
    \begin{itemize}
    \item Če velja \(\typejudgment{x}{A}\), velja \(\typejudgment{\iota_{1}(a)}{A + B}\).
    \item Če velja \(\typejudgment{y}{B}\), velja \(\typejudgment{\iota_{2}(b)}{A + B}\).
    \end{itemize}
  \item \emph{Produkt} \(A \times B\), za katerega veljajo sklepi:
    \begin{itemize}
    \item Če velja \(\typejudgment{t}{A \times B}\), velja \(\typejudgment{\pi_{1}(t)}{A}\).
    \item Če velja \(\typejudgment{t}{A \times B}\), velja \(\typejudgment{\pi_{2}(t)}{B}\).
    \item Če veljata \(\typejudgment{x}{A}\) in \(\typejudgment{y}{B}\), velja
      \(\typejudgment{(x, y)}{A \times B}\).
    \item Za vsak \(\typejudgment{a}{A}\) in \(\typejudgment{b}{B}\) velja
      \(\pi_{1}(a, b) \doteq a\) in \(\pi_{2}(a, b) \doteq b \).
    \end{itemize}
  \item \emph{Funkcijski tip} \(A \to B\), za katerega veljajo sklepi:
    \begin{itemize}
    \item Če veljata \(\typejudgment{x}{A}\) in \(\defun{f}{A}{B}\), velja
      \(\typejudgment{f(x)}{B}\).
    \item Če pod predpostavko \(\typejudgment{x}{A}\) velja \(\typejudgment{b(x)}{B}\),
      velja \(\typejudgment{\lambda x.b(x)}{A \to B}\).
    \item Za vsak \(\typejudgment{a}{A}\) velja enakost \((\lambda x.b(x))(a) \doteq b(a)\).
    \item Za vsako funkcijo \(\defun{f}{A}{B}\) velja enakost \(\lambda x.f(x) \doteq f\).
    \end{itemize}
  \end{itemize}

  S tipi ne želimo predstaviti samo matematičnih objektov,
  temveč tudi matematične trditve, kar bomo dosegli preko tako imenovane interpretacije
  \emph{izjave kot tipi} (ang. \emph{propositions as types}). V tej interpretaciji
  vsaki trditvi dodelimo tip, elementi katerega so dokazila, da ta trditev velja.
  Tako konstrukcija elementa \(\typejudgment{p}{P}\) postane konstrukcija dokazila
  veljavnosti trditve \(P\), kar pa ustreza dokazu trditve \(P\). Če si sedaj zopet
  ogledamo konstrukcije koprodukta, produkta in funkcijskih tipov, vidimo, da ustrezajo
  disjunkciji, konjunkciji in implikaciji med trditvami.

  Kadar lahko pod predpostavko \(\typejudgment{x}{A}\) tvorimo tip \(B(x)\), takrat \(B\)
  imenujemo \emph{odvisen tip} oz. \emph{družina tipov nad A}.
  Za \(A\) tip in \(B\) družino tipov nad \(A\) definiramo še dva sestavljena tipa:

  \begin{enumerate}
  \item \emph{Odvisna vsota} \(\sumtype{x}{A}{B(x)}\), za katero veljajo sklepi:
    \begin{itemize}
    \item Če velja \(\typejudgment{t}{\sumtype{x}{A}{B(x)}}\),
      velja \(\typejudgment{\pi_{1}(t)}{A}\).
    \item Če velja \(\typejudgment{t}{\sumtype{x}{A}{B(x)}}\),
      velja \(\typejudgment{\pi_{2}(t)}{B(\pi_{1}(t))}\).
    \item Če veljata \(\typejudgment{x}{A}\) in \(\typejudgment{y}{B(x)}\), velja
      \(\typejudgment{(x, y)}{\sumtype{x}{A}{B(x)}}\).
    \item Za vsak \(\typejudgment{a}{A}\) in \(\typejudgment{b}{B(a)}\) velja
      \(\pi_{1}(a, b) \doteq a\) in \(\pi_{2}(a, b) \doteq b \).
    \end{itemize}
  \item \emph{Odvisni produkt} \(\pitype{x}{A}{B(x)}\), za katerega veljajo sklepi:
    \begin{itemize}
    \item Če veljata \(\typejudgment{x}{A}\) in
      \(\typejudgment{y}{\pitype{x}{A}{B(x)}}\), velja
      \(\typejudgment{f(x)}{B(x)}\).
    \item Če pod predpostavko \(\typejudgment{x}{A}\) velja \(\typejudgment{b(x)}{B(x)}\),
      velja \(\typejudgment{\lambda x.b(x)}{\pitype{x}{A}{B(x)}}\).
    \item Za vsak \(\typejudgment{a}{A}\) velja enakost \((\lambda x.b(x))(a) \doteq b(a)\).
    \item Za vsako element \(\typejudgment{f}{\pitype{x}{A}{B(x)}}\)
      velja enakost \(\lambda x.f(x) \doteq f\).
    \end{itemize}
  \end{enumerate}

  V interpretaciji \emph{izjave kot tipi} odvisna vsota ustreza
  eksistenčni kvantifikaciji,
  odvisni produkt pa univerzalni kvantifikaciji.
  Če je namreč \(A\) tip in \(P\) družina tipov nad \(A\), lahko tedaj
  element \(\typejudgment{p}{P(a)}\) razumemo kot dokazilo, da
  \emph{predikat P velja za element a}. Tako lahko par
  \(\typejudgment{(a, p)}{\sumtype{x}{A}{P(x)}}\) razumemo kot dokazilo, da
  \emph{obstaja a, za katerega velja P(a)}, funkcijo
  \(\typejudgment{f}{\pitype{x}{A}{P(x)}}\) pa kot prireditev, ki vsakemu elementu
  \(\typejudgment{x}{A}\) priredi dokazilo \(\typejudgment{f(x)}{P(x)}\), torej kot
  dokazilo, da \emph{za vsak x velja P(x)}.
\end{section}

\begin{section}{Sintetična homotopska teorija}
  V tem poglavju bomo predstavili, kako lahko odvisno teorijo tipov uprabimo za razvoj
  tako imenovane \emph{sintetične homotopske teorije}. V tej interpretaciji
  bomo tipe razumeli kot \emph{prostore}, njihove elemente kot \emph{točke}, družine
  tipov \(B\) nad \(A\) pa kot \emph{vlaknenja nad A}, kjer vsaki točki \(x\)
  prostora \(A\) priredimo prostor \(B(x)\), ki \emph{leži nad x}. Odvisno vsoto
  \(\sumtype{x}{A}{B(x)}\) lahko tako razumemo kot \emph{totalni prostor vlaknenja B},
  odvisni produkt \(\pitype{x}{A}{B(x)}\) pa kot \emph{prostor prerezov vlaknenja B}.

  Za homotopsko interpretacijo
  odvisne teorije tipov bo ključna vključitev \emph{tipa identifikacij}: za vsaki točki
  \(\typejudgment{x, y}{A}\) bomo definiramo tip \(\typename{Id}(x, y)\),
  ki ga bomo razumeli
  kot \emph{tip poti med točkama x in y v prostoru A}. Identifikacije med točkami
  lahko razširimo do \emph{homotopije med funkcijami}: za funkciji \(\defun{f, g}{A}{B}\)
  definiramo tip
  \[f \sim g := \pitype{x}{A}{\typename{Id}(f(x), g(x))}.\]
  Elementi \(f \sim g\) so torej dokazila, da za vsako točko \(\typejudgment{x}{A}\) obstaja
  pot med točkama \(f(x)\) in \(g(x)\), kar ustreza klasični definiciji homotopije.

  Za funkcijo \(\defun{f}{A}{B}\) lahko sedaj definiramo \emph{tip prerezov}
  \[\sect(f) := \sumtype{g}{B \to A}{(f \circ g \sim id)}\] in \emph{tip retrakcij}
  \[\ret(f) := \sumtype{g}{B \to A}{(g \circ f \sim id)}.\]
  Preko pojmov prereza in retrakcije lahko definiramo \emph{ekvivalenco tipov},
  ki ustreza homotopski ekvivalenci
  med prostori. Pravimo, da je funkcija \(f\) \emph{ekvivalenca}, če ima tako prerez, kot
  retrakcijo, torej če obstaja element tipa
  \[\isequiv(f) := \sect(f) \times \ret(f),\] za tipa \(A\) in \(B\) pa pravimo, da sta
  \emph{ekvivalentna}, če obstaja ekvivalenca med njima, torej element tipa
  \[A \simeq B := \sumtype{f}{A \to B}{\isequiv(f)}.\]
  Elementi \(A \simeq B\) so torej peterice \((f, g, H, h, K)\), kjer je \(H\) homotopija,
  ki dokazuje, da je \(g\) prerez \(f\), \(K\) pa homotopija, ki dokazuje, da je \(h\)
  retrakcija \(f\).

  V definiciji ekvivalence smo zahtevali, da ima dana funkcija \(f\)
  ločen levi in desni inverz, lahko pa bi lahko zahtevali tudi,
  da ima obojestranski inverz, kar lahko izrazimo s tipom
  \[\isinv(f) := \sumtype{g}{B \to A}{((f \circ g \sim id) \times (g \circ f \sim id))},\]
  ki ga imenujemo \emph{tip inverzov funkcije f}. Če obstaja element tega tipa, tedaj
  za funkcijo \(f\)
  pravimo, da je \emph{obrnljiva}. Čeprav se na prvi pogled zdi,
  da je pojem inverza močnejši od pojma ekvivalence, bomo pokazali, da sta pravzaprav
  logično ekvivalentna; dokaz je povsem analogen dokazu v algebri, kjer lahko pokažemo,
  da ima element z levim in desnim inverzom enoličen obojestranski inverz, le da dokaz
  poteka na nivoju homotopij, namesto enakosti.

  Kljub temu pa med obrnljivostjo in ekvivalenco obstaja
  razlika: pokazali bomo, da pod predpostavko obstoja tako imenovanih
  \emph{višjih induktivnih tipov} obstajajo funkcije \(f\), za katere ne obstaja
  ekvivalenca \(\isequiv(f) \simeq \isinv(f)\). Da bi to razliko bolje razumeli, bomo vpeljali
  pojem \emph{kontraktibilnosti tipov}: pravimo, da je tip \(A\) \emph{kontraktibilen},
  če obstaja element tipa
  \[\iscontr(A) := \sumtype{c}{A}{\pitype{x}{A}{\typename{Id}(c, x)}},\]
  kar pove, da v tipu \(A\) obstaja odlikovana točka \(c\), s katero so identificirane
  vse ostale točke.
  Preko kontraktibilnosti definiramo še sledeč tip:
  \[\isprop(A) := A \to \iscontr(A).\]
  Kontraktibilne tipe lahko razumemo kot tipe, ki imajo \emph{natanko eno točko},
  tipe, za katere velja \(\isprop\), pa kot tipe, ki imajo \emph{največ eno točko}, saj
  zanje velja, da so kontraktibilni, čim vsebujejo točko.

  Pokazali bomo, da ekvivalence ohranjajo lastnost \(\isprop\), torej, da
  če velja \(A \simeq B\) in \(\isprop(A)\), da tedaj velja tudi \(\isprop(B)\), za tem pa še
  skicirali dokaz trditve \(\isprop(\isequiv(f))\) za poljubno funkcijo \(f\).
  Dokaz trditve je precej tehničen, njena vsebina pa ni nepričakovana:
  če ima funkcija levi in desni inverz, sta ta do homotopije natanko enolično določena.
  Trditev pa nam pravzaprav pove še več,
  poleg enoličnosti levega in desnega inverza nam da tudi enoličnost homotopij, ki to
  potrjujeta.

  Sedaj je dokaz neekvivalence med \(\isequiv(f)\) in \(\isinv(f)\) na dlani: če bi taka
  ekvivalenca obstajala, bi ohranjala lastnost \(\isprop\) tipa \(\isequiv(f)\), torej
  moramo le še poiskati funkcijo \(f\), za katero \(\isprop(\isinv(f))\) ne velja.
  Izkaže pa se, da v teoriji tipov, s katero smo delali do sedaj, obstoja take funkcije
  ne moremo ne dokazati, ne ovreči, pokazali pa bomo, da ga lahko dokažemo v določeni
  razširitvi. V namen sintetične homotopske teorije našo teorijo tipov
  pogosto razširimo z različnimi \emph{višjimi induktivnimi tipi}: to so tipi za katere ne
  predpostavimo samo sklepov za konstrukcijo točk, temveč tudi sklepe za konstrukcijo
  identifikacij. Predstavili bomo enega od najpomembnejših primerov višjih induktivnih
  tipov, \emph{sfere} \(\mathbb{S}^{n}\), in pokazali da identitetna funkcija
  na \(\mathbb{S}^{1}\) ne zadošča \(\isprop(\isinv(\mathrm{id}_{\mathbb{S}^{1}}))\). Za
  podrobnosti o konstrukcijah z višjimi induktivnimi tipi se bomo sklicali na
  \cite{hottbook} in \cite{ljungström2024symmetric}
\end{section}

\begin{section}{Karakterizacija obrnljivosti}
  V zadnjem poglavju bomo predstavili novo karakterizacijo
  tipa obrnljivih funkcij med tipoma \(A\) in \(B\):
  \begin{equation}
    \label{mjau}
      \typename{invertible-map}(A, B) := \sumtype{f}{A \to B}{\isinv(f)}.
  \end{equation}
  Dokazali bomo, da velja ekvivalenca
  \[\typename{invertible-map}(A, B) \simeq \sumtype{e}{A \simeq B}{\typename{Id}(e, e)},\]
  formalizacija česar je tudi dostopna v knjižnici formalizirane matematike z dokazovalnim
  pomočnikom Agda \cite{agda-unimath}.
  Na koncu bomo vpeljali še tako imenovane \emph{svetove} \(\mathcal{U}\), tipe, katerih elementi so
  tipi, in aksiom \emph{univalence}, ki opisuje tipe identifikacij na svetovih.
  Preko univerzalnih lastnosti sfer, aksioma univalence in trditve \ref{mjau}
  bomo dokazali sledečo ekvivalenco:
  \[\sumtype{A}{\mathcal{U}}{\sumtype{B}{\mathcal{U}}{(\typename{invertible-map}(A, B))}} \simeq
    (\mathbb{S}^{2} \to \mathcal{U}).\]
\end{section}
\printbibliography
\end{document}
%%% Local Variables:
%%% mode: latex
%%% TeX-master: t
%%% End:
