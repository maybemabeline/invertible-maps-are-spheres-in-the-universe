\documentclass[mat1]{fmfdelo}
% \documentclass[fin1, tisk]{fmfdelo}
% Če pobrišete možnost tisk, bodo povezave obarvane,
% na začetku pa ne bo praznih strani po naslovu, …

%%%%%%%%%%%%%%%%%%%%%%%%%%%%%%%%%%%%%%%%%%%%%%%%%%%%%%%%%%%%%%%%%%%%%%%%%%%%%%%
% METAPODATKI
%%%%%%%%%%%%%%%%%%%%%%%%%%%%%%%%%%%%%%%%%%%%%%%%%%%%%%%%%%%%%%%%%%%%%%%%%%%%%%%

% - vaše ime
\avtor{Mabel Najdovski}

% - naslov dela v slovenščini
\naslov{Obrnljive funkcije so sfere v svetu}

% - naslov dela v angleščini
\title{Invertible maps are spheres in the universe}

% - ime mentorja/mentorice s polnim nazivom:
%   - doc.~dr.~Ime Priimek
%   - izr.~prof.~dr.~Ime Priimek
%   - prof.~dr.~Ime Priimek
%   za druge variante uporabite ustrezne ukaze
\mentor{prof.~dr.~Andrej~Bauer}
\somentor{asist.~-~razisk.~PhD.~Egbert~Marteen~Rijke}

% - leto diplome
\letnica{2024}

% - povzetek v slovenščini
%   V povzetku na kratko opišite vsebinske rezultate dela. Sem ne sodi razlaga
%   organizacije dela, torej v katerem razdelku je kaj, pač pa le opis vsebine.
\povzetek{V povzetku na kratko opišemo vsebinske rezultate dela. Sem ne sodi
razlaga organizacije dela -- v katerem poglavju/razdelku je kaj, pač pa le opis
vsebine.}

% - povzetek v angleščini
\abstract{Prevod slovenskega povzetka v angleščino.}

% - klasifikacijske oznake, ločene z vejicami
%   Oznake, ki opisujejo področje dela, so dostopne na strani https://www.ams.org/msc/
\klasifikacija{74B05, 65N99}

% - ključne besede, ki nastopajo v delu, ločene s \sep
\kljucnebesede{naravni logaritem\sep nenaravni algoritem}

% - angleški prevod ključnih besed
\keywords{natural logarithm\sep unnatural algorithm} % angleški prevod ključnih besed

% - angleško-slovenski slovar strokovnih izrazov
\slovar{
  \geslo{dependent type theory}{odvisna teorija tipov}
  \geslo{type}{tip}
  \geslo{judgment}{sodba}
  \geslo{rule of inference}{pravilo sklepanja}
  \geslo{judgmental equality}{sodbena enakost}
  \geslo{context}{kontekst}
  \geslo{function type}{funkcijski tip}
  \geslo{dependent type}{odvisen tip}
  \geslo{type family}{družina tipov}
  \geslo{dependent product}{odvisni produkt}
  \geslo{dependent function}{odvisna funkcija}
  \geslo{dependent sum}{odvisna vsota}
  \geslo{induction principle}{princip indukcije}
  \geslo{propositions as types}{izjave kot tipi}
  \geslo{witness}{priča}
  \geslo{identity type}{tip identifikacij}
  \geslo{identification elimination, path induction}{eliminacija identifikacij}
  \geslo{constructor}{konstruktor}
  \geslo{transport along identification}{transport vzdolž identifikacije}
  \geslo{synthetic homotopy theory}{sintetična homotopska teorija}
  \geslo{homotopy type theory}{homotopska teorija tipov}
  \geslo{homotopy}{homotopija}
  \geslo{function extensionality}{funkcijska ekstenzionalnost}
  \geslo{invertible}{obrnljiva}
  \geslo{section}{prerez}
  \geslo{retraction}{retrakcija}
  \geslo{equivalence}{ekvivalenca}
  \geslo{mere proposition, h-proposition}{propozicija}
  \geslo{contractible}{kontraktibilen}
  \geslo{center of contraction}{središče kontrakcije}
  \geslo{contraction}{kontrakcija}
  \geslo{h-set}{množica}
  \geslo{higher inductive type}{višji induktivni tip}
  \geslo{circle}{krožnica}
  \geslo{subtype}{podtip}
  \geslo{universe}{svet}
  \geslo{univalence}{univalenca}
  \geslo{universal property}{univerzalna lastnost}
  \geslo{sphere}{sfera}
}

% - ime datoteke z viri (vključno s končnico .bib), če uporabljate BibTeX
\literatura{literatura.bib}

%%%%%%%%%%%%%%%%%%%%%%%%%%%%%%%%%%%%%%%%%%%%%%%%%%%%%%%%%%%%%%%%%%%%%%%%%%%%%%%
% DODATNE DEFINICIJE
%%%%%%%%%%%%%%%%%%%%%%%%%%%%%%%%%%%%%%%%%%%%%%%%%%%%%%%%%%%%%%%%%%%%%%%%%%%%%%%

\usepackage{amsmath}
\usepackage{comment}
\usepackage{quiver}
\usepackage{hyperref}
\usepackage{mathtools}

\newenvironment{konstrukcija}{\paragraph{Konstrukcija:}}{\hfill$\square$}

\newcommand{\N}{\mathbb N}
\newcommand{\Z}{\mathbb Z}

\newcommand{\typejudgment}[2]{#1 : #2}
\newcommand{\defun}[3]{#1 \,\colon #2 \to #3}
\newcommand{\depfun}[3]{#1 \,\colon #2 \to_{\ast} #3}
\newcommand{\eqtype}[3]{#1 \,\colon #2 = #3}
\newcommand{\homotopy}[3]{#1 \,\colon #2 \sim #3}
\newcommand{\dequiv}[3]{#1 \,\colon #2 \simeq #3}
%\newcommand{\sumtype}[3]{\sum\left( \typejudgment{#1}{#2} \right) #3}
%\newcommand{\pitype}[3]{\prod \left( \typejudgment{#1}{#2} \right) #3}
\newcommand{\sumtype}[3]{\textstyle \sum_{\left( \typejudgment{#1}{#2} \right)} #3}
\newcommand{\pitype}[3]{\textstyle \prod_{\left( \typejudgment{#1}{#2} \right)} #3}
\newcommand{\typename}[1]{\textnormal{\textsf{#1}}}
\newcommand{\emptytype}{\textnormal{\textbf{0}}}
\newcommand{\unittype}{\textnormal{\textbf{1}}}
\newcommand{\type}[1]{#1\ \typename{type}}
\newcommand{\proj}[1]{\typename{pr}_{#1}}
\newcommand{\natzero}{\typename{0}}
\newcommand{\natsucc}{\typename{S}}
\newcommand{\id}[1]{\typename{id}_{#1}}
\newcommand{\Id}[1]{\typename{Id}_{#1}}
\newcommand{\ap}[1]{\typename{ap}_{#1}\,}
\newcommand{\htpyarr}[1]{\xrightarrow[\sim]{#1}}

%%%Common types
\newcommand{\isinv}{\typename{inverse}}
\newcommand{\iscohinv}{\typename{is-coh-invertible}}
\newcommand{\isequiv}{\typename{is-equiv}}
\newcommand{\iscontr}{\typename{is-contr}}
\newcommand{\isprop}{\typename{is-prop}}
\newcommand{\refl}[1]{\typename{refl}_{#1}}
\newcommand{\map}{\typename{map}}
\newcommand{\sect}{\typename{sec}}
\newcommand{\ret}{\typename{ret}}
\newcommand{\tot}{\typename{tot}}
\newcommand{\transp}[1]{\typename{tr}_{#1}}
\newcommand{\fib}[2]{\typename{fib}_{#1}\left(#2\right)}
\newcommand{\sphere}[1]{\mathbb{S}^{#1}}
\newcommand{\krog}{\sphere{1}}
\newcommand{\base}{\typename{base}}
\newcommand{\lop}{\typename{loop}}
\newcommand{\surf}{\typename{surf}}
\newcommand{\ind}[1]{\typename{ind}_{#1}}
\newcommand{\invmap}{\typename{inv-map}}

%%% Local Variables:
%%% mode: latex
%%% TeX-master: "./Najdovski-27191110-2023"
%%% End:


%%%%%%%%%%%%%%%%%%%%%%%%%%%%%%%%%%%%%%%%%%%%%%%%%%%%%%%%%%%%%%%%%%%%%%%%%%%%%%%
% ZAČETEK VSEBINE
%%%%%%%%%%%%%%%%%%%%%%%%%%%%%%%%%%%%%%%%%%%%%%%%%%%%%%%%%%%%%%%%%%%%%%%%%%%%%%%

\begin{document}

\section{Osnove Martin-Löfove odvisne teorije tipov}

Sledeč (TODO cite intro to hott) bomo v tem poglavju predstavili osnovne koncepte
Martin-Löfove odvisne teorije tipov. Zaradi velike razsežnosti in dokajšnje tehničnosti
formalne predstavitve teorije tipov se ne bomo do potankosti spustili v njene podrobnosti,
vendar bomo koncepte predstavili v nekoliko neformalnem slogu, osredotočajoč se na
analogije z bolj standardnimi matematičnimi koncepti iz formalizma teorije množic.

\subsection{Sodbe in pravila sklepanja}

Osnovni gradniki teorije tipov so \emph{tipi}, ki jih bomo označevali z velikimi tiskanimi
črkami, kot so \(A, B\) in \(C\), ter njihovi \emph{elementi}, ki jih bomo označevali
z malimi tiskanimi črkami, kot so \(x, y, z\) in \(a, b, c\). Vsak element \(x\) ima
natanko določen tip \(A\), kar izrazimo s tako imenovano \emph{sodbo}
\(\typejudgment{x}{A}\), ki jo preberemo kot ``\(x\) je tipa \(A\)''.
Ta je v mnogih pogledih podobna relaciji \(x \in A\) iz teorije množic, vendar se od
nje razlikuje v nekaj pomembnih pogledih.

Prvič, v teoriji množic je relacija \(\in\) v določenem smislu \emph{globalna};
vsak matematični objekt je kodiran kot določena množica in za poljubni množici \(A\) in
\(B\) se lahko vprašamo, ali velja \(A \in B\). Tako je na primer vprašanje, ali za element
\(g\) grupe \(\Z/_{5}\) velja \(g \in \mathcal{C}^{1}(\mathbb{R})\) povsem smiselno,
vendar tudi povsem v nasprotju z matematično prakso.
V nasprotju s tem so v teoriji tipov elementi nerazdružljivi od svojih tipov; če velja
sodba \(\typejudgment{x}{A}\), potem vprašanja ``Ali velja \(\typejudgment{x}{B}\)?''
sploh ne moremo tvoriti.

Drugič, sodba \(\typejudgment{x}{A}\) ni trditev, ki bi jo lahko dokazali, temveč je
konstrukcija, do katere lahko pridemo po nekem končnem zaporedju \emph{pravil sklepanja}.
To so sintaktična pravila, ki nam omogočijo, da iz določenega nabora
sodb tvorimo nove sodbe. Primer pravila sklepanja bi bilo pravilo, ki pravi, da
lahko iz sodb \(\typejudgment{x}{A}\) in \(\defun{f}{A}{B}\) sklepamo, da velja
sodba \(\typejudgment{f(x)}{B}\), kar predstavlja uporabo funkcije \(f\)
na elementu \(x\).

Poleg sodb oblike \(\typejudgment{x}{A}\) v teoriji tipov obstajajo še tri vrste sodb.
Prvič, obstaja sodba \(\type{A}\), s katero izrazimo, da je tip \(A\) dobro tvorjen.
Primer pravila sklepanja s to sodbo bi bilo
pravilo, ki pravi, da lahko iz sodb \(\type{A}\) in \(\type{B}\) sklepamo, da velja
sodba \(\type{A \to B}\), kar predstavlja tvoritev tipa funkcij med tipoma \(A\) in \(B\).
Drugič, obstajata še sodbi \(\typejudgment{x \doteq y}{A}\) in \(\type{A \doteq B}\),
s katerima izražamo \emph{sodbeno enakost}. Za obe vrsti sodbene enakosti veljajo pravila
sklepanja, preko katerih postaneta ekvivalenčni relaciji,
veljajo pa še določena substitucijska
pravila, ki omogočajo, da lahko v poljubnem izrazu elemente in tipe nadomeščamo s sodbeno
enakimi elementi in tipi. Ker bomo kasneji spoznali še drugo vrsto enakosti, je vredno
poudariti, da je sodbena enakost zelo stroga oblika enakosti. Sodbeno enake izraze lahko
zato razumemo kot različne zapise za \emph{isti} element.

Pomemben element teorije tipov, ki smo ga do sedaj izpuščali, je dejstvo,
da vsaka sodba nastopa v določenem \emph{konstekstu}. Konteksti so končni seznami
deklariranih spremenljivk \(\typejudgment{x_k}{A_{k}}\), ki jih sodbe v tem kontekstu
lahko vsebujejo, razumemo pa jih lahko tudi kot nabor predpostavk, pod katerimi določena
sodba velja.  To, da je \(\mathcal{J}\) sodba v kontekstu \(\Gamma\), zapišemo kot \(\Gamma \vdash \mathcal{J}\).
Primer pravila sklepanja s sodbo v konstekstu bi bilo pravilo, ki pravi,
da lahko iz sodbe
\(\typejudgment{x}{A} \vdash \typejudgment{b}{B}\) sklepamo, da velja
\(\emptyset \vdash \typejudgment{\lambda x.b(x)}{A \to B}\), kar predstavlja tvoritev
anonimne funkcije z vezavo spremenljivke \(x\).

Razpravo povzamemo v sledeči vzajemno rekurzivni definiciji sodb in kontekstov.

\begin{definicija}
  V Martin-Löfovi odvisni teoriji tipov obstajajo štiri vrste sodb:
\begin{enumerate}
\item \(A\) je tip v kontekstu \(\Gamma\), kar izrazimo z \[\Gamma \vdash \type{A}.\]
\item \(A\) in \(B\) sta \emph{sodbeno enaka tipa} tipa v kontekstu \(\Gamma\),
  kar izrazimo z \[\Gamma \vdash \type{A \doteq B}.\]
\item \(x\) je element tipa \(A\) v kontekstu \(\Gamma\), kar izrazimo z
  \[\Gamma \vdash \typejudgment{x}{A}.\]
\item \(x\) in \(y\) sta \emph{sodbeno enaka elementa} v kontekstu \(\Gamma\), kar izrazimo z
  \[\Gamma \vdash \typejudgment{x \doteq y}{A}.\]
\end{enumerate}
\emph{Kontekst} je induktivno definiran seznam sodb, podan s praviloma:
\begin{enumerate}
\item Obstaja prazen kontekst \(\emptyset\).
\item Če je \(\Gamma\) kontekst in lahko prek obstoječih pravil sklepanja izpeljemo sodbo
  \(\Gamma \vdash \type{A}\), je tedaj \(\Gamma, \typejudgment{x}{A}\) kontekst.
\end{enumerate}
\end{definicija}
Ekspliciten zapis kontekstov in sodb bomo od sedaj deloma izpuščali in o
njih govorili v bolj naravnem jeziku. Ko bomo torej trditve začeli stavki kot je
``Naj bo \(A\) tip in \(x\) element \(A\)...'', smo pravzaprav predpostavili, da velja
sodba \(\type{A}\) in da delamo v kontestku \(\Gamma = \typejudgment{x}{A}\).

S temi definicijami lahko za konec tega podpoglavja predstavimo še specifikaciji
funkcijskih in produktnih tipov. Velik del pravil sklepanja za funkcijske tipe smo že
videli na primerih, tu pa jih še dopolnimo in zberemo v definicijo.

\begin{definicija}
  Naj bosta \(A\) in \(B\) tipa. Tedaj lahko tvorimo \emph{funkcijski tip} \(A \to B\),
  za katerega veljajo sledeča pravila sklepanja:
  \begin{enumerate}
  \item Če veljata \(\typejudgment{x}{A}\) in \(\defun{f}{A}{B}\), velja
    \(\typejudgment{f(x)}{B}\).
  \item Če pod predpostavko \(\typejudgment{x}{A}\) velja \(\typejudgment{b(x)}{B}\),
    velja \(\typejudgment{\lambda x.b(x)}{A \to B}\).
  \item Za vsak \(\typejudgment{a}{A}\) velja enakost \((\lambda x.b(x))(a) \doteq b(a)\).
  \item Za vsak \(\defun{f}{A}{B}\) velja enakost \(\lambda x.f(x) \doteq f\).
  \end{enumerate}
  Elemente funkcijskega tipa imenujemo \emph{funkcije}.
\end{definicija}
\begin{definicija}
  Naj bosta \(A\) in \(B\) tipa. Tedaj lahko tvorimo \emph{produktni tip} \(A \times B\),
  za katerega veljajo sledeča pravila sklepanja.
  \begin{enumerate}
  \item Če velja \(\typejudgment{t}{A \times B}\), veljata \(\typejudgment{\proj{1}(t)}{A}\) in
    \(\typejudgment{\proj{2}(t)}{B}\).
  \item Če veljata \(\typejudgment{x}{A}\) in \(\typejudgment{y}{B}\), velja
    \(\typejudgment{(x, y)}{A \times B}\).
  \item Za vsaka \(\typejudgment{x}{A}\) \(\typejudgment{y}{B}\) veljata enakosti
    \(\proj{1}(a, b) \doteq a\) in \(\proj{2}(a, b) \doteq b\).
  \item Za vsak \(\typejudgment{t}{A \times B}\) velja enakost
    \((\proj{1}(t), \proj{2}(t)) \doteq t\).
  \end{enumerate}
  Elemente produktnega tipa imenujemo \emph{pari}.
\end{definicija}

\subsection{Odvisne vsote in odvisni produkti}
Pomemben element \emph{odvisne} teorije tipov je dejstvo, da lahko sodbe \(\type{A}\)
tvorimo v poljubnem kontekstu, to pa pomeni da lahko izrazimo tudi sodbe
oblike \(\typejudgment{x}{A} \vdash \type{B(x)}\).
V tem primeru pravimo, da je \(B\) \emph{odvisen tip},
oziroma, da je \emph{B družina tipov nad A}. Odvisni tipi nam bodo omogočili, da izrazimo
tipe, parametrizirane s spremenljivko nekega drugega tipa.

Za delo z odvisnimi tipi vpeljemo še dve pomembni konstrukciji. Prvič, definiramo
\emph{odvisni produkt tipov A in B}, elementi katerega so predpisi, ki vsakemu elementu
\(\typejudgment{x}{A}\) priredijo element v pripadajočem tipu \(B(x)\).
Če na \(B\) gledamo kot družino tipov
nad \(A\), lahko na elemente odvisnega produkta gledamo kot na \emph{prereze družine B},
pogosto pa jih imenujemo tudi \emph{odvisne funkcije} ali \emph{družine elementov B,
parametrizirane z A}. Drugič, definiramo \emph{odvisno vsoto tipov A in B}, elementi
katere so pari \((x, y)\), kjer je \(\typejudgment{x}{A}\) in \(\typejudgment{y}{B(x)}\).
Z odvisno vsoto tako zberemo celotno družino tipov \(B\) v skupen tip,
na pare \((x, y)\) pa lahko gledemo kot na elemente \(\typejudgment{y}{B(x)}\), označene z
elementom \(\typejudgment{x}{A}\), nad katerim ležijo.
Drugače pogledano lahko na pare \((x, y)\) gledamo tudi kot na elemente
\(\typejudgment{x}{A}\), opremljene z dodatno strukturo \(y\) iz
pripadajočega tipa \(B(x)\).

\begin{definicija}
  Naj bo \(A\) tip in \(B\) družina tipov nad \(A\). Tedaj lahko tvorimo tip
  \[\pitype{x}{A}{B(x)},\] imenovan \emph{odvisni produkt tipov A in B}.
  Zanj veljajo sledeča pravila sklepanja:
  \begin{enumerate}
  \item Če veljata \(\typejudgment{x}{A}\) in
    \(\typejudgment{f}{\pitype{x}{A}{B(x)}}\), velja
    \(\typejudgment{f(x)}{B(x)}\).
  \item Če pod predpostavko \(\typejudgment{x}{A}\) velja \(\typejudgment{b(x)}{B(x)}\),
    velja \(\typejudgment{\lambda x.b(x)}{\pitype{x}{A}{B(x)}}\).
  \item Za vsak \(\typejudgment{a}{A}\) velja enakost \((\lambda x.b(x))(a) \doteq b(a)\).
  \item Za vsak \(\typejudgment{f}{\pitype{x}{A}{B(x)}}\) velja enakost \(\lambda x.f(x) \doteq f\).
  \end{enumerate}
\end{definicija}

\begin{definicija}
  Naj bo \(A\) tip in \(B\) družina tipov nad \(A\). Tedaj lahko tvorimo tip
  \[\sumtype{x}{A}{B(x)},\] imenovan \emph{odvisna vsota tipov A in B}.
  Zanj veljajo sledeča pravila sklepanja
  \begin{enumerate}
  \item Če velja \(\typejudgment{t}{\sumtype{x}{A}{B(x)}}\), veljata
    \(\typejudgment{\proj{1}(t)}{A}\) in
    \(\typejudgment{\proj{2}(t)}{B(\proj{1}(t))}\).
  \item Če veljata \(\typejudgment{x}{A}\) in \(\typejudgment{y}{B(x)}\), velja
    \(\typejudgment{(x, y)}{\sumtype{x}{A}{B(x)}}\).
  \item Za vsaka \(\typejudgment{x}{A}\) \(\typejudgment{y}{B(x)}\) veljata enakosti
    \(\proj{1}(a, b) \doteq a\) in \(\proj{2}(a, b) \doteq b\).
  \item Za vsak \(\typejudgment{t}{\sumtype{x}{A}{B(x)}}\) velja enakost
    \((\proj{1}(t), \proj{2}(t)) \doteq t\).
  \end{enumerate}
\end{definicija}
Opazimo lahko podobnost med pravili sklepanja za odvisne produkte in
funkcije ter za odvisne vsote in produkte. Ta podobnost ni naključje, saj če
je \(A\) tip in \(B\) tip, neodvisen od \(A\), lahko nanj kljub temu gledamo kot na tip,
\emph{trivialno} odvisen od \(A\). V tem primeru veljata enakosti
\(\sumtype{x}{A}{B} \doteq A \times B\) in \(\pitype{x}{A}{B} \doteq A \to B\).
\pagebreak

\begin{primer}
  S pomočjo odvisnih produktov lahko podamo specifikacijo tipa
  \emph{naravnih števil} \(\N\), za katera veljajo sledeča pravila sklepanja.
  \begin{enumerate}
  \item Obstaja naravno število \(\typejudgment{\typename{0}}{\N}\).
  \item Če velja \(\typejudgment{n}{\N}\),
    velja \(\typejudgment{\natsucc(n)}{\N}\).
  \item Velja sledeč \emph{princip indukcije}. Naj bo \(B\) družina tipov nad \(\N\) in
    denimo, da veljata \(\typejudgment{b_{\natzero}}{B(\natzero)}\) in
    \(\typejudgment{b_{\natsucc}}{\pitype{n}{\N}{(B(n) \to B(\natsucc(n)))}}\).

    Tedaj velja
    \(\typejudgment{\typename{ind}_{\N}(b_{\natzero}, b_{\natsucc})}{\pitype{n}{\N}{B(n)}}\).
  \item Veljata enakosti
    \(\typename{ind}(b_{\natzero}, b_{\natsucc}, \natzero) \doteq b_{\natzero}\) in
    \(\typename{ind}(b_{\natzero}, b_{\natsucc}, \natsucc(n)) \doteq
      b_{\natsucc}(n, \typename{ind}(b_{\natzero}, b_{\natsucc}, n)\).
  \end{enumerate}
  Princip indukcije nam pove, da če želimo za vsako naravno število \(\typejudgment{n}{\N}\)
  konstruirati element tipa \(B(n)\), tedaj zadošča, da konstruiramo element tipa \(B(\natzero)\)
  in pa da znamo za vsak \(\typejudgment{n}{\N}\) iz elementov tipa \(B(n)\) konstruirati elemente
  tipa \(B(\natsucc(n))\), v čemer lahko prepoznamo podobnost z običajnim principom indukcije v
  teoriji množic.
\end{primer}

\subsection{Interpretacija \emph{izjave kot tipi}}

S tipi ne želimo predstaviti samo matematičnih objektov, temveč tudi matematične trditve,
kar bomo dosegli preko tako imenovane interpretacije \emph{izjave kot tipi}
(ang. \emph{propositions as types}). V tej interpretaciji vsaki trditvi \(P\) dodelimo
tip z istim imenom, elementi katerega so dokazi oz. \emph{priče}, da ta trditev velja.
Konstrukcija
elementa \(\typejudgment{p}{P}\) tako postane konstrukcija priče \(p\), ki potrjuje
veljavnost trditve \(P\), to pa ustreza dokazu trditve \(P\).

Če si sedaj iz vidika izjav kot tipov zopet ogledamo pravila sklepanja za produktne in
funkcijske tipe, lahko v njih prepoznamo
pravila sklepanja za konjunkcijo in implikacijo. Dokaz trditve \(P \wedge Q\) je namreč sestavljen
iz para dokazov posameznih trditev \(P\) in \(Q\), dokaz implikacije \(P \Rightarrow Q\) pa lahko razumemo
kot predpis, ki iz dokazov \(P\) konstruira dokaze \(Q\). Uporaba funkcije \(P \to Q\)
tako ustreza pravilu \emph{modus ponens}, ki pravi, da lahko iz veljavnosti izjav \(P\) in \(P \Rightarrow Q\)
sklepamo veljavnost izjave \(Q\).

Denimo sedaj, da je \(A\) tip in \(P\) družina tipov nad \(A\), vsak od katerih predstavlja določeno
izjavo. Tako družino izjav imenujemo tudi \emph{predikat na A}.
Tedaj lahko logično interpretacijo podamo tudi odvisni vsoti in odvisnemu produktu;
ustrezata namreč eksistenčni in univerzalni kvantifikaciji. Elementi odvisnega vsote
\(\sumtype{x}{A}{P(x)}\) so pari \((x, p)\), kjer je \(p\) dokaz \(P(x)\), zato lahko nanje gledamo
kot na priče veljavnosti \(\exists \typejudgment{x}{A}. P(x)\), elementi odvisnega produkta
\(\pitype{x}{A}{P(x)}\) pa so predpisi, ki vsakemu elementu \(\typejudgment{x}{A}\) priredijo
dokaz izjave \(P(x)\), zato lahko nanje gledamo kot na priče veljavnosti
\(\forall \typejudgment{x}{A}.P(x)\).

Definicije predikatov in formulacije trditev, ki jih bomo predstavili v preostanku dela, bomo
izrazili v jeziku interpretacije \emph{izjave kot tipi}, kljub temu pa se lahko eksplicitnemu
zapisu prič pogosto izognemo. V dokazih trditev bomo tako pogosto uporabljali naravni
jezik sklepanja, vedno pa pravzaprav konstruiramo element določenega tipa, ki predstavlja trditev,
ki jo dokazujemo.

%%% Local Variables:
%%% mode: latex
%%% TeX-master: "../Najdovski-27191110-2023"
%%% End:

\section{Sintetična homotopska teorija}


\subsection{Identifikacije in homotopije}
Osnovna motivacija za vpeljavo tipa identifikacij je dejstvo, da je pojem sodbene enakosti
\emph{prestrog} in da prek nje pogosto ne moremo identificirati vseh izrazov, ki bi
jih želeli. Poleg tega si v skladu z interpretacijo \emph{izjave kot tipi} želimo tip, elementi
katerega bi predstavljali dokaze enakosti med dvema elementoma.

Pomankljivost sodbene enakosti si lahko ogledamo na primeru
komutativnosti naravnih števil.

\begin{primer}
  Seštevanje naravnih števil bomo definirali kot funkcijo
  \[\defun{\typename{sum}}{\N}{(\N \to \N)},\] njeno evaluacijo \(\typename{sum}(m, n)\)
  pa bomo kot običajno pisali kot \(m + n\).
  Ker je tip \(\N \to (\N \to \N)\) po opombi \ref{neodvisna-vsota} sodbeno enak tipu
  \(\pitype{n}{\N}{\N \to \N}\), lahko njegove elemente konstruiramo po indukcijskem
  principu naravnih števil.
  Izpuščajoč nekaj podrobnosti to pomeni, da želimo vrednost izraza \(m + n\) definirati z
  indukcijo na \(n\), torej moramo v baznem primeru podati vrednost izraza \(m + \natzero\),
  v primeru koraka pa moramo pod predpostavko, da poznamo vrednost izraza \(m + n\),
  podati vrednost izraza \(m + \natsucc(n)\). Naravno je,
  da izberemo \(m + \natzero :\doteq m\) in \(m + \natsucc(n) :\doteq\natsucc(m + n)\).
  Za poljubni \emph{fiksni} naravni števili, recimo
  \(\typename{4}\) in \(\typename{5}\), lahko sedaj z uporabo pravil sklepanja za naravna
  števila pod točko \ref{naturals} dokažemo, da velja
  \(\typename{4} + \typename{5} \doteq \typename{5} + \typename{4}\). Če pa sta po drugi strani
  \(m\) in \(n\)
  \emph{spremenljivki} tipa \(\N\), tedaj zaradi asimetričnosti definicije seštevanja
  enakosti \(n + m \doteq m + n\) ne moremo
  dokazati. Potrebujemo tip \emph{identifikacij} \(\Id{\N}\), z uporabo katerega bi
  lahko z indukcijo pokazali, da velja \[\pitype{m, n}{\N}{\Id{\N}(n + m, m + n)}.\]
\end{primer}

\begin{definicija}
  Naj bo \(A\) tip in \(\typejudgment{a, x}{A}\). Tedaj lahko tvorimo tip
  \(\Id{A}(a, x)\),
  imenovan \emph{tip identifikacij med elementoma a in x}. Zanj veljajo sledeča pravila
  sklepanja:
  \begin{enumerate}
  \item Za vsak element \(\typejudgment{a}{A}\) obstaja identifikacija
    \(\typejudgment{\refl{a}}{\Id{A}(a, a)}\).
  \item Velja sledeč princip \emph{eliminacije identifikacij}.
    Naj bo \(\typejudgment{a}{A}\), \(B\) družina tipov, indeksirana z
    \(\typejudgment{x}{A}\) ter \(\typejudgment{p}{\Id{A}(a, x)}\) in denimo, da velja
    \(\typejudgment{r(a)}{B(a, \refl{a})}\). Tedaj velja
    \[\typejudgment{\typename{ind}_{\Id{A}}(a, r(a))}
      {\pitype{x}{A}{\pitype{p}{\Id{A}(a, x)}{B(x, p)}}}\]
  \item Za vsak \(\typejudgment{a}{A}\) velja enakost
    \(\typename{ind}_{\Id{A}}(a, r(a))(a, \refl{a}) \doteq r(a)\).
  \end{enumerate}
  Tip identifikacij \(\Id{A}(a, x)\) bomo od sedaj pisali kar kot \(a = x\) in za boljšo
  berljivost izpuščali ekspliciten zapis tipa \(A\). Kadar je \(p\) identifikacija
  med \(a\) in \(x\), bomo torej pisali \(\eqtype{p}{a}{x}\).

  Konstruktor \(\eqtype{\refl{a}}{a}{a}\)
  nam zagotavlja, da za vsak element \(\typejudgment{a}{A}\) obstaja kanonična
  identifikacija elementa samega s sabo. Princip eliminacije identifikacij nato zatrdi,
  da če želimo za vsako identifikacijo \(\eqtype{p}{a}{x}\) konstruirati element tipa
  \(B(x, p)\), tedaj zadošča, da konstruiramo le element tipa \(B(a, \refl{a})\).
  Pravimo tudi, da je družina tipov \(a = x\) \emph{induktivno definirana} s
  konstruktorjem \(\refl{a}\).
.\end{definicija}

  Resnična motivacija za takšno definicijo tipa identifikacij sega pregloboko, da bi se
  vanjo spustili v tem delu, povemo pa lahko, da lahko preko nje dokažemo vse željene
  lastnosti, ki bi jih od identifikacij pričakovali. Nekaj od teh vključuje:
\begin{enumerate}
\item Velja \emph{simetričnost}: za vsaka elementa \(\typejudgment{x, y}{A}\) obstaja
  funkcija \[\defun{\typename{sym}}{(x = y)}{(y = x)}.\] Identifikacijo \(\typename{sym}(p)\)
  imenujemo \emph{inverz} identifikacije \(p\) in ga pogosto označimo z \(p^{-1}\).
\item Velja \emph{tranzitivnost}: za vsake elemente \(\typejudgment{x, y, z}{A}\) obstaja
  funkcija \[\defun{\typename{concat}}{(x = y)}{((y = z) \to (x = z))},\] ki jo imenujemo
  \emph{konkatenacija identifikacij}, \(\typename{concat}(p, q)\) pa pogosto označimo
  z \(p \cdot q\).
\item Funkcije ohranjajo identifikacije: za vsaka elementa \(\typejudgment{x, y}{A}\)
  in vsako funkcijo \(\defun{f}{A}{B}\) obstaja funkcija
  \[\defun{\typename{ap}_{f}}{(x = y)}{(f(x) = f(y))},\]
  ki jo imenujemo \emph{aplikacija} funkcije \(f\) na identifikacije v \(A\).
\item Družine tipov spoštujejo identifikacije: za vsaka elementa
  \(\typejudgment{x, y}{A}\), identifikacijo \(\eqtype{p}{x}{y}\) in družino tipov
  \(B\) nad \(A\) obstaja funkcija
  \[\defun{\typename{tr}_{B}(p)}{B(x)}{B(y)}.\] To pomeni, da lahko elemente tipa
  \(B(x)\) \emph{transportiramo} vzdolž identifikacije \(\eqtype{p}{x}{y}\), da dobimo
  elemente \(B(y)\).
  V interpretaciji \emph{izjave kot tipi} to pomeni, da če je \(P\) predikat na \(A\)
  in velja identifikacija \(x = y\), da tedaj \(P(x)\) velja natanko tedaj kot \(P(y)\),
  saj lahko transport uporabimo tudi na \(p^{-1}\).
\end{enumerate}
Konstrukcije naštetih funkcij lahko poiščemo v 5.~poglavju (TODO cite intro to hott), vse
izmed njih pa zahtevajo le preprosto uporabo eliminacije identifikacij.
(TODO grupoidala struktura tipov, interpretacija identifikacij kot poti)

Izkaže se, da smo s tako definicijo identifikacij zelo omejeni v konstruiranju identifikacij
med funkcijami. Namesto tega lahko identifikacije med funkcijami nadomestimo s
t.i.~\emph{homotopijami}, ki v mnogih primerih zadoščajo, konstruiramo pa jih veliko lažje.
Homotopija med funkcijami zatrdi, da se ti ujemata na vsakem elementu domene.
\begin{definicija}
  Naj bosta \(A\) in \(B\) tipa ter \(f\) in \(g\) funkciji tipa \(A \to B\). Definiramo tip
  \[f \sim g := \pitype{x}{A}{f(x) = g(x)},\]
  imenovan \emph{tip homotopij med f in g}, njegove elemente pa imenujemo \emph{homotopije}.
\end{definicija}

\begin{opomba}
  Na kratko komentiramo o podobnosti med homotopijami v teoriji tipov in homotopijami v
  klasični topologiji. Naj bosta \(X\) in \(Y\) topološka prostora in
  \(\defun{f, g}{X}{Y}\) zvezni
  funkciji. Z \(I\) označimo interval \([0, 1]\). Homotopijo med funkcijama \(f\) in \(g\)
  tedaj definiramo kot zvezno funkcijo \(\defun{H}{I \times X}{Y}\), za katero velja
  \(H(0, x) = f(x)\) in \(H(1, x) = g(x)\). Ker pa je \(I\) lokalno kompakten prostor, so
  zvezne funkcije \(\defun{H}{I \times X}{Y}\) v bijektivni korespondenci z zveznimi funkcijami
  \(\defun{\hat{H}}{X}{\mathcal{C}(I, Y)}\), kjer je \(\mathcal{C}(I, Y)\) prostor zveznih funkcij med \(I\) in
  \(Y\), opremljen s kompaktno-odprto topologijo. Zvezne funkcije \(\defun{\alpha}{I}{Y}\)
  ustrezajo potem v \(Y\) med točkama \(\alpha(0)\) in \(\alpha(1)\). Če je torej \(H\)
  homotopija med \(f\) in \(g\), za pripadajočo funkcijo \(\hat{H}\) velja, da je
  \(\hat{H}(x)\) pot med točkama \(f(x)\) in \(g(x)\) za vsak \(x \in X\). To je analogno
  definiciji homotopij med funkcijama \(f\) in \(g\) v teoriji tipov, ki vsaki točki
  \(\typejudgment{x}{X}\) priredijo identifikacijo \(f(x) = g(x)\).
\end{opomba}

Operacije na identifikacijah lahko po elementih razširimo do operacij na homotopijah,
za delo s homotopijami
pa bomo potrebovali še operaciji (TODO whiskering?), s katerima lahko homotopije z leve ali
z desne razširimo s funkcijo.

\begin{trditev}
  Naj bosta \(A\) in \(B\) tipa in \(\defun{f, g, h}{A}{B}\) funkcije. Denimo, da obstajata
  homotopiji \(\homotopy{H}{f}{g}\) in \(\homotopy{K}{g}{h}\).

  Tedaj obstajata homotopiji \(\homotopy{H^{-1}}{g}{f}\) in \(\homotopy{H \cdot K}{f}{h}\).
\end{trditev}

\begin{dokaz}
  Če homotopiji H in K evaluiramo na elementu \(\typejudgment{x}{A}\), dobimo identifikaciji
  \(\eqtype{H(x)}{f(x)}{g(x)}\) in \(\eqtype{K(x)}{g(x)}{h(x)}\). Homotopijo \(H^{-1}\) lahko
  torej definiramo kot \(\lambda x. H(x)^{-1}\), homotopijo \(H \cdot K\) pa kot \(\lambda x.(H(x) \cdot K(x))\).
\end{dokaz}

\begin{trditev}
  Naj bodo \(A, B, C\) in \(D\) tipi ter \(\defun{h}{A}{B}\), \(\defun{f, g}{B}{C}\) in
  \(\defun{k}{C}{D}\) funkcije. Denimo, da obstaja homotopija \(\homotopy{H}{f}{g}\).
  Tedaj obstajata homotopiji \(\homotopy{Hh}{f \circ h}{g \circ h}\) in
  \(\homotopy{kH}{k \circ f}{k \circ g}\).
\end{trditev}

\begin{dokaz}
  Homotopijo \(H\) zopet evaluiramo na elementu \(\typejudgment{y}{B}\) in dobimo
  identifikacijo \(\eqtype{H(y)}{f(y)}{g(y)}\). Če sedaj nanjo apliciramo funkcijo \(h\),
  dobimo identifikacijo \(\eqtype{\ap{h}(H(y))}{h(f(y))}{h(g(y))}\). Homotopijo \(kH\) lahko
  torej definiramo kot
  \(\lambda y. \typename{ap}_{k}(H(y))\).

  Naj bo sedaj \(\typejudgment{x}{A}\). Homotopijo \(H\) lahko tedaj evaluiramo tudi
  na elementu \(h(x)\) in tako dobimo identifikacijo
  \(\eqtype{H(h(x))}{f(h(x))}{g(h(x))}\). Homotopijo \(Hh\) lahko torej definiramo kot
  \(\lambda x.H(h(x))\).
\end{dokaz}

\subsection{Obrnljivost in ekvivalenca}


\begin{definicija}
  Za funkcijo \(f\) pravimo, da je \emph{obrnljiva} oziroma, da \emph{ima inverz},
  če obstaja element tipa
  \[\isinv(f) := \sumtype{g}{B \to A}{(f \circ g \sim id) \times (g \circ f \sim id)},\]
  imenovanega \emph{tip inverzov funkcije f}.

  Funkcije, pripadajoče elementom \(\isinv(f)\), imenujemo \emph{inverzi funkcije f}.
\end{definicija}

\begin{definicija}
  Za funkcijo \(f\) pravimo, da \emph{ima prerez}, če obstaja element tipa
  \[\sect(f) := \sumtype{g}{B \to A}{f \circ g \sim id},\]
  imenovanega \emph{tip prerezov funkcije f}.

  Funkcije, pripadajoče elementom tipa \(\sect f\) imenujemo \emph{prerezi funkcije f}.
  Za funkcijo \(f\) pravimo, da \emph{ima retrakcijo}, če obstaja element tipa
  \[\ret(f) := \sumtype{g}{B \to A}{g \circ f \sim id},\]
  imenovanega \emph{tip retrakcij funkcije f}.

  Funkcije, pripadajoče elementom tipa \(\ret f\) imenujemo \emph{retrakcije funkcije f}.
\end{definicija}

\begin{definicija}
  Pravimo, da je funkcija \(f\) \emph{ekvivalenca}, če ima tako prerez kot retrakcijo,
  torej, če obstaja element tipa \[\isequiv(f) := \sect(f) \times \ret(f).\]
  Pravimo, da je tip \(A\) \emph{ekvivalenten} tipu \(B\), če obstaja ekvivalenca med
  njima, torej element tipa \(A \simeq B := \sumtype{f}{A \to B}{\isequiv(f)}\). Funkcijo,
  pripadajočo elementu \(\dequiv{e}{A}{B}\), označimo z \(\map e\).
\end{definicija}

\subsection{Osnovne lastnosti}

V definiciji obrnljivosti smo zahtevali, da ima funkcija obojestranski inverz, v
definiciji ekvivalence pa smo zahtevali le, da ima ločen levi in desni inverz.
To bi nas lahko
napeljalo k prepričanju, da je pojem obrnljivosti močnejši od pojma ekvivalence, vendar
spodnja trditev pokaže da sta pravzaprav logično ekvivalentna. Pokazali bomo, da lahko
prerez (ali simetrično, retrakcijo) ekvivalence \(f\) vedno izboljšamo do inverza,
kar pokaže, da je \(f\) tudi obrnljiva.

\begin{trditev}
  \label{inv-of-equiv}
  Funkcija je ekvivalenca natanko tedaj, ko je obrnljiva.
\end{trditev}

\begin{dokaz}
  Denimo, da je funkcija \(f\) obrnljiva. Tedaj lahko njen inverz podamo tako kot njen
  prerez, kot njeno retrakcijo, kar pokaže, da je ekvivalenca.

  Obratno denimo, da je funkcija \(f\) ekvivalenca. Podan imamo njen prerez \(s\) s
  homotopijo \(\homotopy{H}{f \circ s}{id}\) in njeno retrakcijo \(r\) s homotopijo
  \(\homotopy{K}{r \circ f}{id}\), s katerimi lahko konstruiramo homotopijo tipa
  \(s \circ f \sim id\) po sledečem izračunu:
  \[s f \overset{K^{-1}sf}{\sim} r f s f \overset{rHf}{\sim} r f \overset{K}{\sim} id. \qedhere\]
\end{dokaz}

\begin{definicija}
  Naj bo \(\typejudgment{f}{\pitype{x}{A}{\left(Bx \to Cx\right)}}\) družina funkcij.
  Definiramo funkcijo
  \(\typejudgment{\tot(f)}{\sumtype{x}{A}{Bx}} \to \sumtype{x}{A}{Cx}\) s predpisom
  \(\tot(f)(x, y) = (x, f(x)(y))\).
\end{definicija}

\begin{trditev}
  Naj bo \(\typejudgment{f}{\pitype{x}{A}{\left(Bx \to Cx\right)}}\) družina funkcij
  in denimo, da je \(f(x)\) ekvivalenca za vsak \(\typejudgment{x}{A}\). Tedaj je
  ekvivalenca tudi funkcija \(\tot(f)\).
\end{trditev}

\begin{dokaz}
  Ker je funkcija \(f(x)\) ekvivalenca za vsak \(\typejudgment{x}{A}\),
  lahko tvorimo družino funkcij
  \(\typejudgment{s}{\pitype{x}{A}{\left(Cx \to Bx\right)}}\), kjer je \(s(x)\)
  prerez funkcije \(f(x)\).
  Trdimo, da je tedaj \(\tot(s)\) prerez funkcije \(\tot(f)\), saj
  za vsak \(\typejudgment{(x, y)}{\sumtype{x}{A}{Cx}}\) velja enakost
  \[\tot(f)(\tot(s)(x, y)) = \tot(f)(x, s(x)(y)) = (x, f(x)(s(x)(y))) = (x, y).\]
  Analogno lahko konstruiramo retrakcijo funkcije \(\tot(f)\), kar zaključi dokaz.
\end{dokaz}

Zgornja trditev je pomembna, saj nam omogoči, da konstrukcijo ekvivalence med odvisnima
vsotama z istim baznim tipom poenostavimo na konstrukcijo družine ekvivalenc,
kar je pogosto veliko lažje. Uporabljali ga bomo v obliki sledeče posledice:

\begin{posledica}
  \label{equiv-tot}
  Naj bo \(A\) tip, \(B\) in \(C\) družini tipov nad \(A\) in denimo, da velja
  \(Bx \simeq Cx\) za vsak \(\typejudgment{x}{A}\).
  Tedaj velja \(\sumtype{x}{A}{Bx} \simeq \sumtype{x}{A}{Cx}\).
\end{posledica}

%%% Local Variables:
%%% mode: latex
%%% TeX-master: "../Najdovski-27191110-2023"
%%% End:

\section{Karakterizacija obrnljivosti}

\begin{definicija}
  \emph{Prosta zanka} na tipu \(A\) je sestavljena iz točke \(\typejudgment{a}{A}\) in
  identifikacije \(a = a\). Tip vseh prostih zank na tipu \(A\) označimo s
  \[\typename{free-loop}(A) := \sumtype{x}{A}{x = x}.\]
\end{definicija}

\begin{izrek}
  Tip prostih zank na tipu \(A \simeq B\) je ekvivalenten tipu obrnljivih funkcij
  med \(A\) in \(B\).
\end{izrek}

\begin{dokaz}
  Želimo konstruirati ekvivalenco med tipom \(\sumtype{e}{A \simeq B}{e = e}\) in tipom
  \(\sumtype{f}{A \to B}{\isinv(f)}\).
  Ker je \(\isequiv\) predikat in za vsako funkcijo \(f\) obstaja funkcija
  \(\typename{is-invertible}(f) \to \typename{is-equiv}(f)\), najprej opazimo, da po
  trditvi \ref{full-subtype} velja ekvivalenca
  \[\sumtype{f}{A \to B}{\isinv(f)} \simeq
    \sumtype{e}{A \simeq B}{\isinv(\map e).}
  \]
  (TODO define \map) Po trditvi (TODO equiv-tot)
  torej zadošča pokazati, da za vsako ekvivalenco \(\dequiv{e}{A}{B}\) obstaja ekvivalenca
  \[(e = e) \simeq \isinv(\map \, e).\]
  Oglejmo si tip \[\isinv(\map e) = \sumtype{g}{B \to A}{(\map e \circ g \sim id) \times (g \circ \map e \sim id)}.\] Po
  asociativnosti tipa odvisne vsote je ta ekvivalenten tipu
  \begin{align*}
    & \sumtype{H}{\sumtype{g}{B \to A}{\map e \circ g \sim id}}{(\map H \circ \map e \sim id)} = \\
    & \sumtype{H}{\typename{section}(\map e)}{(\map H \circ \map e \sim id)},
  \end{align*}
  ker pa imajo po trditvi TODO ekvivalence kontraktibilen tip prerezov, po trditvi
  (TODO kontraktibilen bazni prostor) velja še ekvivalenca
  \[\sumtype{H}{\typename{section}(\map e)}{(\map H \circ \map e \sim id)} \simeq
    \left(\typename{sec} e \circ \map e \sim id\right).\]
  Sledi, da velja \(\isinv(\map e) \simeq (\typename{sec} e \circ \map e \sim id)\), dokaz pa
  zaključimo še z zaporedjem ekvivalenc, ki jih argumentiramo spodaj.
  \begin{align*}
    & (\typename{sec} e \circ \map e \sim id) \simeq \\
    & (\map e \circ \typename{sec}e \circ \map e \sim \map e) \simeq \\
    & (\map e \sim \map e) \simeq \\
    & (\map e = \map e) \simeq \\
    & (e = e)
  \end{align*}
  \begin{itemize}
  \item Ker je \(e\) ekvivalenca, je po trditvi TODO ekvivalenca tudi delovanje \(\map e\)
    na homotopije.
  \item Funkcija \(\typename{sec}e\) je prerez funkcije \(\map e\).
  \item TODO funext
  \item Po trditvi \ref{subtype-id} lahko zanko na funkciji \(\map e\) dvignemo do
    zanke na pripadajoči ekvivalenci \(e\).
  \end{itemize}
\end{dokaz}


%%% Local Variables:
%%% mode: latex
%%% TeX-master: "../Najdovski-27191110-2023"
%%% End:


\end{document}

%%% Local Variables:
%%% mode: latex
%%% TeX-master: t
%%% End:
