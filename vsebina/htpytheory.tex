\section{Sintetična homotopska teorija}
Martin-Löfovo odvisno teorijo tipov bomo v tem poglavju uporabili za razvoj t.i. \emph{sintetične homotopske teorije}. V tej interpretaciji na tipe gledamo kot na topološke prostore, na njihove elemente pa kot na točke v teh prostorih. Za to interpretacijo je ključno opažanje, da se identifikacije med dvema elementoma v mnogih pogledih obnašajo kot poti med točkama v pripadajočih prostorih. Med dvema elementoma lahko namreč obstaja več različnih identifikacij, ki jih ni moč identificirati, to pa je analogno dejstvu, da lahko med dvema točkama v topološkem prostoru obstaja več poti, med katerimi ne obstaja zvezna deformacija. Preko funkcij \(\typename{sym}\) in \(\typename{concat}\) lahko identifikacije kot poti invertiramo in stikamo skupaj, funkcija \(\typename{ap}\) pa nam za vsako funkcijo zagotavlja določeno obliko zveznosti, saj zatrdi, da vsaka funkcija identifikacije slika v identifikacije. Preko identifikacij lahko tako brez omembe odprtih množic ali zveznosti pridemo do mnogih rezultatov iz klasične homotopske teorije.

V tem delu se bomo predvsem posvetili problemu definicije \emph{ekvivalence tipov}, ki je analogna homotopski ekvivalenci med pripadajočima prostoroma. Kot v definiciji tipa  identifikacij želimo definirati tip, elementi katerega bi predstavljali dokaze ekvivalence med dvema tipoma. Predstavili bomo dve možnosti za definicijo takega tipa in pokazali, da je nekoliko presenetljivo ena od njiju veliko manj primerna od druge. Da bi najbolje predstavili razliko med definicijama bomo vpeljali pojem \emph{kontraktibilnosti tipov}, ki je analogen pojmu kontraktibilnosti topoloških prostorov, in pa tip \(\sphere{1}\), ki je analogen topološki sferi.

\subsection{Homotopije}
Ekvivalenco tipov \(A\) in \(B\) bomo definirali kot obstoj funkcije \(\defun{f}{A}{B}\), ki
je v primernem smislu obrnljiva, za to pa potrebujemo dobro karakterizacijo tipa identifikacij med funkcijami. V obravnavi obrnljivosti bomo namreč potrebovali identifikacije oblike \(f \circ g = \id{A}\).

Izkaže pa se, da smo v možnostih za konstrukcijo identifikacij med funkcijami pogosto zelo omejeni, zato jih bomo nadomestili t.i.~\emph{homotopijami}. Te so v večini primerov zadostne, konstruiramo pa jih veliko lažje. Homotopijo med funkcijama definiramo kot odvisno funkcijo, ki zatrdi, da se ti ujemata na vsakem elementu domene.
\begin{definicija}
  Naj bosta \(A\) in \(B\) tipa ter \(f\) in \(g\) funkciji tipa \(A \to B\). Definiramo tip
  \[f \sim g := \pitype{x}{A}{f(x) = g(x)},\]
  imenovan \emph{tip homotopij med f in g}, njegove elemente pa imenujemo \emph{homotopije}.
\end{definicija}

\begin{opomba}
  Na kratko komentiramo o podobnosti med homotopijami v teoriji tipov in homotopijami v
  topologiji. Naj bosta \(X\) in \(Y\) topološka prostora in
  \(\defun{f, g}{X}{Y}\) zvezni
  funkciji. Z \(I\) označimo interval \([0, 1]\). Homotopijo med funkcijama \(f\) in \(g\)
  tedaj običajno definiramo kot zvezno funkcijo \(\defun{H}{I \times X}{Y}\), za katero velja
  \(H(0, x) = f(x)\) in \(H(1, x) = g(x)\). Ker pa je \(I\) lokalno kompakten prostor, so
  zvezne funkcije \(\defun{H}{I \times X}{Y}\) v bijektivni korespondenci z zveznimi funkcijami
  \(\defun{\hat{H}}{X}{\mathcal{C}(I, Y)}\), kjer je \(\mathcal{C}(I, Y)\) prostor zveznih funkcij med \(I\) in
  \(Y\), opremljen s kompaktno-odprto topologijo. Zvezne funkcije \(\defun{\alpha}{I}{Y}\)
  ustrezajo potem v \(Y\) med točkama \(\alpha(0)\) in \(\alpha(1)\). Če je torej \(H\)
  homotopija med \(f\) in \(g\), za pripadajočo funkcijo \(\hat{H}\) velja, da je
  \(\hat{H}(x)\) pot med točkama \(f(x)\) in \(g(x)\) za vsak \(x \in X\). To je analogno
  definiciji homotopij med funkcijama \(f\) in \(g\) v teoriji tipov, ki vsaki točki
  \(\typejudgment{x}{X}\) priredijo identifikacijo \(f(x) = g(x)\).
\end{opomba}

Operacije na identifikacijah lahko po elementih razširimo do operacij na homotopijah,
za delo s homotopijami
pa bomo potrebovali še operaciji (TODO whiskering?), s katerima lahko homotopije z leve ali
z desne razširimo s funkcijo.

\begin{trditev}
  Naj bosta \(A\) in \(B\) tipa in \(\defun{f, g, h}{A}{B}\) funkcije. Denimo, da obstajata
  homotopiji \(\homotopy{H}{f}{g}\) in \(\homotopy{K}{g}{h}\).

  Tedaj obstajata homotopiji \(\homotopy{H^{-1}}{g}{f}\) in \(\homotopy{H \cdot K}{f}{h}\).
\end{trditev}

\begin{dokaz}
  Če homotopiji H in K evaluiramo na elementu \(\typejudgment{x}{A}\), dobimo identifikaciji
  \(\eqtype{H(x)}{f(x)}{g(x)}\) in \(\eqtype{K(x)}{g(x)}{h(x)}\). Homotopijo \(H^{-1}\) lahko
  torej definiramo kot \(\lambda x. H(x)^{-1}\), homotopijo \(H \cdot K\) pa kot \(\lambda x.(H(x) \cdot K(x))\).
\end{dokaz}

\begin{trditev}
  Naj bodo \(A, B, C\) in \(D\) tipi ter \(\defun{h}{A}{B}\), \(\defun{f, g}{B}{C}\) in
  \(\defun{k}{C}{D}\) funkcije. Denimo, da obstaja homotopija \(\homotopy{H}{f}{g}\).
  Tedaj obstajata homotopiji \(\homotopy{Hh}{f \circ h}{g \circ h}\) in
  \(\homotopy{kH}{k \circ f}{k \circ g}\).
\end{trditev}

\begin{dokaz}
  Homotopijo \(H\) zopet evaluiramo na elementu \(\typejudgment{y}{B}\) in dobimo
  identifikacijo \(\eqtype{H(y)}{f(y)}{g(y)}\). Če sedaj nanjo apliciramo funkcijo \(h\),
  dobimo identifikacijo \(\eqtype{\ap{h}(H(y))}{h(f(y))}{h(g(y))}\). Homotopijo \(kH\) lahko
  torej definiramo kot
  \(\lambda y. \typename{ap}_{h}(H(y))\).

  Naj bo sedaj \(\typejudgment{x}{A}\). Homotopijo \(H\) lahko tedaj evaluiramo tudi
  na elementu \(h(x)\) in tako dobimo identifikacijo
  \(\eqtype{H(h(x))}{f(h(x))}{g(h(x))}\). Homotopijo \(Hh\) lahko torej definiramo kot
  \(\lambda x.H(h(x))\).
\end{dokaz}

Dejstvo, da se pojma enakosti in homotopije med funkcijami lahko razlikujeta, je lahko na prvi pogled presenetljivo, ne pomeni pa nujno, da med seboj homotopne vendar različne funkcije lahko konstruiramo. Martin-Löfova odvisna teorija tipov preprosto ne vsebuje pravil sklepanja, preko katerih bi lahko iz \(f \sim g\) izpeljali \(f = g\). In res, obstaja metateoretični izrek o Martin-Löfovi odvisni teoriji tipov, ki pove, da obstoja takih funkcij ne moremo ne dokazati, ne ovreči. Z drugimi besedami, njihov obstoj je od teorije neodvisen. Martin-Löfovo teorijo tipov zato pogosto razširimo z aksiomom \emph{funkcijske ekstenzionalnosti}, ki pove, da se homotopnost funkcij sklada z enakostjo funkcij.

Ta aksiom bomo večkrat uporabili, njegova točna formulacija pa je zunaj obsega tega dela. Kot s karakterizacijo identifikacij v odvisnih vsotah, bomo enakost med funkcijami prosto izmenjevali z homotopijo med funkcijami.

Vredno je poudariti, da je bila karakterizacija enakosti v odvisnih vsotah \emph{izrek} Martin-Löfove teorije tipov, ki ga sicer nismo dokazali, ta karakterizacija enakosti v funkcijskih tipih pa je \emph{aksiom}, zato njegova uporaba nosi nekoliko več teže. Ko bomo enakost nadomeščali s homotopijo, bomo zato to tudi izpostavili.

\subsection{Obrnljivost in ekvivalenca}
Opremljeni s homotopijami lahko sedaj obrnljivost funkcije \(\defun{f}{A}{B}\) definiramo kot obstoj funkcije \(\defun{g}{B}{A}\), za katero velja \(f \circ g \sim \id{}\) in
\(g \circ f \sim \id{}\). Dokazi obrnljivosti \(f\) so torej trojice \((g, H, K)\), kjer sta \(H\) in \(K\) omenjeni homotopiji.
\begin{definicija}
  Za funkcijo \(f\) pravimo, da je \emph{obrnljiva} oziroma, da \emph{ima inverz},
  če obstaja element tipa
  \[\isinv(f) := \sumtype{g}{B \to A}{(f \circ g \sim \id{}) \times (g \circ f \sim \id{})},\]
  imenovanega \emph{tip inverzov funkcije f}.

  Funkcije, pripadajoče elementom \(\isinv(f)\), imenujemo \emph{inverzi funkcije f}.
\end{definicija}

Definiramo lahko tudi tip desnih inverzov, imenovanih prerezi, in pa tip levih inverzov, imenovanih retrakcije.
\begin{definicija}
  Za funkcijo \(f\) pravimo, da \emph{ima prerez}, če obstaja element tipa
  \[\sect(f) := \sumtype{g}{B \to A}{f \circ g \sim \id{}},\]
  imenovanega \emph{tip prerezov funkcije f}.
  Funkcije, pripadajoče elementom tipa \(\sect(f)\) imenujemo \emph{prerezi funkcije f}.

  Za funkcijo \(f\) pravimo, da \emph{ima retrakcijo}, če obstaja element tipa
  \[\ret(f) := \sumtype{g}{B \to A}{g \circ f \sim \id{}},\]
  imenovanega \emph{tip retrakcij funkcije f}.
  Funkcije, pripadajoče elementom tipa \(\ret(f)\) imenujemo \emph{retrakcije funkcije f}.
\end{definicija}

Naša alternativna definicija za obrnljivost bo pojem \emph{ekvivalence}. Pravimo, da je funkcija \(f\) ekvivalenca, če obstajata (potencialno različni) funkciji \(g\) in \(h\), za kateri obstajata homotopiji \(f \circ g \sim \id{}\) in \(h \circ f \sim \id{}\).

\begin{definicija}
  Pravimo, da je funkcija \(f\) \emph{ekvivalenca}, če ima tako prerez kot retrakcijo,
  torej, če obstaja element tipa \[\isequiv(f) := \sect(f) \times \ret(f).\]
  Pravimo, da je tip \(A\) \emph{ekvivalenten} tipu \(B\), če obstaja ekvivalenca med
  njima, torej element tipa \(A \simeq B := \sumtype{f}{A \to B}{\isequiv(f)}\).
\end{definicija}

V definiciji obrnljivosti smo zahtevali, da ima funkcija obojestranski inverz, v
definiciji ekvivalence pa smo zahtevali le, da ima ločen levi in desni inverz.
To bi nas lahko
napeljalo k prepričanju, da je pojem obrnljivosti močnejši od pojma ekvivalence, vendar
spodnja trditev pokaže da sta pravzaprav logično ekvivalentna. Pokazali bomo, da lahko
prerez (ali simetrično, retrakcijo) ekvivalence \(f\) vedno izboljšamo do inverza,
kar pokaže, da je \(f\) tudi obrnljiva.

Dokaz je preprost in v mnogih pogledih analogen dokazem v algebri, recimo v teoriji monoidov, kjer velja, da se levi in desni inverz danega elementa vedno ujemata. V našem primeru se levi in desni inverz ne ujemata strogo, lahko pa med njima konstruiramo homotopijo.
\begin{trditev}
  \label{inv-of-equiv}
  Funkcija je ekvivalenca natanko tedaj, ko je obrnljiva.
\end{trditev}

\begin{dokaz}
  Denimo, da je funkcija \(f\) obrnljiva. Tedaj lahko njen inverz podamo tako kot njen
  prerez, kot njeno retrakcijo, kar pokaže, da je ekvivalenca.

  Obratno sedaj denimo, da je funkcija \(f\) ekvivalenca. Podan imamo torej njen prerez
  \(s\) s homotopijo \(\homotopy{H}{f \circ s}{\id{}}\) in njeno retrakcijo \(r\) s homotopijo
  \(\homotopy{K}{r \circ f}{\id{}}\). Homotopijo tipa \(s \circ f \sim \id{}\) lahko tedaj konstruiramo
  po sledečem izračunu:
  \[s f \overset{K^{-1}sf}{\sim} r f s f \overset{rHf}{\sim} r f \overset{K}{\sim} \id{}.\]
  Funkcijo \(s\) lahko tako podamo kot inverz funkcije \(f\).
  Kot omenjeno lahko med funkcijama \(s\) in \(r\) na sledeč način konstruiramo tudi homotopijo:
  \[s \overset{K^{-1}s}{\sim} rfs \overset{rH}{\sim} r. \qedhere\]
\end{dokaz}

Spomnimo se, da v teoriji tipov želimo predikate na tipu \(A\) predstaviti z družinami tipov \(P\) nad \(A\), kjer pravimo, da predikat \(P(x)\) velja, če je pripadajoči tip \(P(x)\) naseljen. V tem podpoglavju smo na tipu \(A \to B\) vpeljali dva predikata, \(\isinv\) in \(\isequiv\), v prejšnji trditvi pa dokazali, da je tip \(\isinv(f)\) naseljen natanko tedaj kot \(\isinv(f)\), torej da sta pripadajoči trditvi ekvivalentni. Vprašamo pa se lahko tudi nekoliko več: ali velja \(\isinv(f) \simeq \isequiv(f)\)?

Drugače pogledano je to, da je tip \(\isinv(f)\) naseljen natanko tedaj kot tip \(\isequiv(f)\) ekvivalentno obstoju funkcij \(\isinv(f) \to \isequiv(f)\) in obratno. Taki funkciji smo v prešnji trditvi implicitno tudi konstruirali, ekvivalenca med tipoma \(\isinv(f)\) in \(\isequiv(f)\) pa bi pomenila, da sta si tudi ti funkciji med seboj tudi inverzni.

Nekoliko presenetljivo se izkaže, da taka ekvivalenca za vsak \(f\) ne more obstajati, do tega pa pride zato, ker imata tipa v določenem smislu drugačno strukturo. Vpeljali bomo namreč lastnost \emph{propozicije} in pokazali, da jo tip \(\isequiv(f)\) izpolnjuje za vsak \(f\), za tip \(\isinv(f)\) pa je to odvisno od strukture tipov \(A\) in \(B\). Nadaljno bomo pokazali, da ekvivalnce ohranjajo lastnost propozicje, torej da za tipa \(A \simeq B\) velja, da tip \(A\) to lastnost izpolnjuje natanko tedaj kot tip \(B\). Posledično ekvivalenca za tiste funkcije \(f\), za katere \(\isinv(f)\) ni propozicija, ne more obstajati.

\subsection{Kontraktibilnost in propozicije}
Na kratko se spomnimo pojma kontraktibilnosti iz klasične topologije. Naj bo torej \(X\) topološki prostor in \(x \in X\). Tedaj pravimo, da je prostor \(X\) \emph{kontraktibilen}, če obstaja homotopija med identiteto na \(X\) in konstantno funkcijo pri \(x\). S tem želimo izraziti, da prostor \(X\) homotopsko gledano nima nobene bistvene strukture in da ima do homotopije natanko samo eno točko, \(x\). V teoriji tipov lahko to lastnost izrazimo na sledeč način.
\begin{definicija}
  Naj bo \(A\) tip. Pravimo, da je tip \(A\) \emph{kontraktibilen}, če obstaja element tipa
  \[\iscontr(A) := \sumtype{c}{A}{\pitype{x}{A}{c = x}}.\]
  Element \(c\) imenujemo \emph{središče kontrakcije}, funkcijo
  \(\typejudgment{C}{\pitype{x}{A}{c = x}}\) pa \emph{kontrakcija}.
\end{definicija}
Če naivno preberemo definicijo, se na prvi pogled zdi, kot da smo s tem zajeli le pojem povezanosti s potmi, saj smo zahtevali obstoj točke \(\typejudgment{c}{A}\), za katero velja, da je vsaka točka \(\typejudgment{x}{A}\) z njo identificirana. Razumeti pa moramo, da to ne ustreza le izboru poti \(c = x\) za vsak \(x\), temveč \emph{zveznemu} izboru takih poti. Tudi v klasični topologiji je obstoj zvezne funkcije \(X \to \mathcal{C}(I, X)\), ki vsaki točki \(x\) priredi pot med izbrano točko \(c\) in \(x\), ekvivalenten kontraktibilnosti \(X\).

Če smo s pojmom kontraktibilnosti želeli izraziti, da ima tip \(A\) \emph{natanko} en element, želimo s pojmom propozicije izraziti, da ima tip \(A\) \emph{največ} en elemnt. Z drugimi besedami, poljubna elementa tipa \(A\) sta identificirana.

\begin{definicija}
  Naj bo \(A\) tip. Pravimo, da je tip \(A\) \emph{propozicija}, če obstaja element tipa
  \[\isprop(A) := \pitype{x, y}{A}{x = y}.\]
\end{definicija}

Da ima tip \(A\) največ eno točko lahko izrazimo tudi na drugačen način, namreč, da če tip \(A\) ima točko, da je ta tedaj tudi edina, torej, da je tip \(A\) tedaj kontraktibilen. Ti formulaciji sta tudi logično ekvivalentni, ob različnih priložnosti pa je katera od njiju lahko bolj prikladna.

\begin{trditev}
  Naj bo \(A\) tip. Tedaj je \(A\) propozicija natanko tedaj, kot velja \(A \to \iscontr(A)\).
\end{trditev}

\begin{dokaz}
  Denimo najprej, da je tip \(A\) propozicija. Konstruirati želimo funkcijo tipa \(A \to \iscontr(A)\), zato denimo, da je \(\typejudgment{a}{A}\). Za središče kontrakcije tipa \(A\) lahko tedaj izberemo kar ta element, konstruirati pa moramo še kontrakcijo, torej funkcijo tipa \(\pitype{x}{A}{a = x}\). To imamo, saj je tip \(A\) propozicija, le eno od krajišč družine identifikacij moramo fiksrati na element \(a\).

  Obratno denimo, da velja \(A \to \iscontr(A)\). Dokazati moramo, da je sta tedaj poljubna elementa tipa \(A\) identificirana, zato denimo, da sta \(\typejudgment{x, y}{A}\). Kateri koli izmed njiju, recimo \(x\), bi tedaj zadoščal, da dobimo kontraktibilnost tipa \(A\). Posledično dobimo par identifikacij tipa \(c = x\) in \(c = y\) za neko središče kontrakcije \(\typejudgment{c}{A}\). Z inverzom in konkatenacijo ju lahko združimo v identifikacijo tipa \(x = y\), kar zaključi dokaz.
\end{dokaz}

Kot omenjeno bomo dokazali, da ekvivalence ohranjajo propozicije. Najprej dokažimo, da ekvivalence ohranjajo tudi kontraktibilnost.

\begin{trditev}
  \label{equiv-preserves-contr}
  Naj bosta \(A\) in \(B\) tipa ter \(A \simeq B\). Tedaj je tip \(A\) kontraktibilen natanko tedaj, kot tip \(B\). Še več, naj bo \(f\) ekvivalenca med \(A\) in \(B\) in \(\typejudgment{c}{A}\) središče kontrakcije tipa \(A\). Tedaj je \(f(c)\) središče kontrakcije tipa \(B\).
\end{trditev}

\begin{dokaz}
  Denimo, da je tip \(A\) kontraktibilen s središčem kontrakcije \(\typejudgment{c}{A}\).

  Ker sta tipa \(A\) in \(B\) ekvivalentna, imamo funkciji \(\defun{f}{A}{B}\) in
  \(\defun{g}{B}{A}\), za kateri velja \(f \circ g \sim \id{B}\). Za središče kontrakcije tipa \(B\) izberemo element \(f(c)\). Dokazati moramo, da za vsak \(\typejudgment{y}{B}\) obstaja identifikacija tipa \(f(c) = y\).

  Naj bo torej \(\typejudgment{y}{B}\). S funkcijo \(g\) ga lahko presikamo v tip \(A\), ker pa je ta kontraktibilen, dobimo identifikacijo tipa \(c = g(y)\). Z aplikacijo funkcije \(f\) jo lahko preslikamo nazaj v tip \(B\) in dobimo identifikacijo tipa \(f(c) = f(g(y))\), ker pa velja tudi \(f(g(y)) = y\), s konkatenacijo dobimo željeno identifikacijo.

  Drugi del trditve velja po konstrukciji, obratno smer pa dobimo povsem simetrično.
\end{dokaz}

\begin{trditev}
  \label{equiv-preserves-prop}
  Naj bosta \(A\) in \(B\) tipa ter \(A \simeq B\). Tedaj je tip \(A\) propozicija natanko tedaj, kot tip \(B\).
\end{trditev}

\begin{dokaz}
  Uporabili bomo alternativno formulacijo propozicij. Denimo torej, da velja \(A \to \iscontr(A)\). Konstruirati moramo funkcijo tipa \(B \to \iscontr(B)\), zato denimo, da velja \(\typejudgment{y}{B}\). Ker sta tipa \(A\) in \(B\) ekvivalentna, lahko \(y\) preslikamo v tip \(A\), od tod pa sledi, da je tip \(A\) naseljen in posledično kontraktibilen. Po prejšnji trditvi je tedaj kontraktibilen tudi tip \(B\).
  Obratno implikacijo dobimo povsem simetrično.
\end{dokaz}

Sedaj želimo pokazati, da je tip \(\isequiv(f)\) res propozicija za vsako funkcijo \(f\), za kar bomo uporabili alternativno formulacijo pojma propozicije. Pod predpostavko, da je funkcija \(f\) ekvivalenca, moramo torej dokazati, da je tip \(\isequiv(f)\) kontraktibilen.

Dokaz sloni na dejstvu, da če je funkcija \(f\) ekvivalenca, da ima tedaj tudi enoličen prerez in enolično retrakcijo, oziroma, da sta tipa \(\sect(f)\) in \(\ret(f)\) tedaj kontraktibilna.
Dokaz te trditve je precej tehničen in zunaj obsega tega dela, zato ga bomo le skicirali. Vredno pa je omeniti, da je predpostavka, da je \(f\) ekvivalenca, tukaj potrebna. V splošnem lahko seveda obstajajo funkcije z mnogo različnimi prerezi ali retrakcijami.

Zadnje dejstvo, ki ga tedaj še potrebujemo, je to, da je produkt kontraktibilnih tipov kontraktibilen, saj je \(\isequiv(f) = \sect(f) \times \ret(f)\). Dokaz te trdi1tve je zelo preprost, sloni pa na karakterizaciji tipa identifikacij v produktih.

\begin{trditev}
  Naj bosta \(A\) in \(B\) tipa ter denimo, da sta kontraktibilna. Tedaj je kontraktibilen tudi tip \(A \times B\).
\end{trditev}

\begin{dokaz}
  Naj bosta \(\typejudgment{a}{A}\) in \(\typejudgment{b}{B}\) središči kontrakcij tipov \(A\) in \(B\) ter \(C_{A}\) in \(C_{B}\) njuni kontrakciji. Tip \(A \times B\) je tedaj kontraktibilen s srdeščem kontrakcije \((a, b)\) in kontrakcijo \(\lambda(x, y).(C_{A}(x), C_{B}(y))\).
\end{dokaz}

Sedaj bomo skicirali dokaz trditve, da imajo ekvivalence res kontraktibilen tip prerezov. Dokaz za retrakcije poteka povsem simetrično.

\begin{trditev}
  \label{contr-sec-of-equiv}
  Naj bo \(\defun{f}{A}{B}\) ekvivalenca med tipoma \(A\) in \(B\). Tedaj je tip prerezov funkcije \(f\) kontraktibilen.
\end{trditev}

\begin{dokaz}
  Ker je funkcija \(f\) ekvivalenca, ima prerez \(\defun{s}{B}{A}\) in retrakcijo \(\defun{r}{B}{A}\) s pripadajočima homotopijama \(H\) in \(K\). Za središče kontrakcije tipa \(\sect(f)\) lahko torej izberemo kar par \((s, H)\), konstruirati pa moramo še kontrakcijo, ki poljuben element \(\sect(f)\) z njim izenači.
  Denimo torej, da je \(\defun{g}{B}{A}\) s homotopijo \(\homotopy{G}{f \circ g}{\id{}}\) element \(\sect(f)\). Tedaj lahko homotopijo med \(s\) in \(g\) konstruiramo po sledečem izračunu:
  \[s \overset{K^{-1}s}{\sim} r \circ f \circ s \overset{rH}{\sim}
    r \overset{rG^{-1}}{\sim} r \circ f \circ g \overset{Kg}{\sim} g.\]
  Kompozitum homotopij označimo z \(L\) in s funkcijsko ekstenzionalnostjo pretvorimo v identifikacijo, ki jo označimo z \(L'\). Po karakterizaciji identifikacij v odvisnih vsotah bi sedaj morali konstruirati še identifikacijo med transportom homotopije \(H\) vzdolž identifikacije \(L'\) in homotopijo \(G\), kjer pa nastopi tehnični del dokaza. Postopamo lahko po sledečih korakih:
  \begin{enumerate}
  \item Poračunamo lahko, da velja enakost \(\transp{}(L', H) = fL^{-1} \cdot H\). Prepričamo se lahko vsaj, da se tipi ujemajo. Da bi ju lahko identificirali, mora biti homotopija \(\transp{}(L', H)\) namreč enakega tipa kot homotopija \(G\), ki je tipa \(f \circ g \sim \id{}\). Spomnimo se še, da je \(L\) homotopija tipa \(s \sim g\). Homotopija, ki smo jo poračunali, je tedaj pravilnega tipa:
    \[f \circ g \overset{fL^{-1}}{\sim} f \circ s \overset{H}{\sim} \id{}.\]

  \item Željeno identifikacijo najprej preoblikujemo v \(fL = H \cdot G^{-1}\). Ker je funkcija \(r\) retrakcija ekvivalence, lahko pokažemo, da je zaradi tega ekvivalenca tudi sama in zato v določenem smislu levo krajšljiva. Konstrukcija željene identifikacije je zato ekvivalentna kontrukciji identifikacije \[rfL = rH \cdot rG^{-1}.\]
    Ker je homotopija \(Ks \cdot L \cdot K^{-1}g\) enaka homotopiji \(rH \cdot rG^{-1}\) po definiciji \(L\), zadošča dokazati, da velja identifikacija \(rfL = Ks \cdot L \cdot K^{-1}g.\)
    Ne nazadnje željeno identifikacijo preoblikujemo v
    \[rfL \cdot Kg = Ks \cdot L.\]

  \item Za konstrukcijo zadnje identifikacije potrebujemo določeno splošnejšo trditev. Naj bodo namreč \(\defun{f, g}{A}{B}\) in \(\defun{h, k}{B}{C}\) funkcije in
    \(\homotopy{H}{f}{g}\) in \(\homotopy{K}{h}{k}\) homotopiji. Tedaj lahko z uporabo funkcijske ekstenzionalnosti in eliminacije identifikacij konstruiramo identifikacijo
    \[hH \cdot Kg = Kf \cdot kH.\]
    Homotopiji \(H\) in \(K\) lahko v homotopijo med funkcijama \(h \circ f\) in \(k \circ g\) združimo na dva različna načina, odvisno od tega ali najprej uporabimo homotopijo \(H\) ali homotopijo \(K\), trditev pa nam pove, da se načina pravzaprav skladata. Mimogrede to združitev imenujemo \emph{horizontalna kompozicija} homotopij \(H\) in \(K\), trditev pa tvori podlago za njeno dobro definiranost.

Željeno identifikacijo lahko nazadnje dobimo tako, da trditev uporabimo na homotopijah \(\homotopy{K}{r \circ f}{\id{}}\) in \(\homotopy{L}{s}{g}\).
\end{enumerate}
\end{dokaz}

To trditev je moč dokazati tudi na bolj eleganten način, vendar za to v tem delu nismo razvili potrebnega ogrodja. Kontraktibilnost smo zato morali pokazati tako rekoč na roke, čemur se v homotopski teoriji tipov ponavadi raje izognemo.

To podpoglavje zaključimo s tem, da diskusijo povzamemo v sledečem izreku:

\begin{izrek}
  \label{is-equiv-prop}
  Naj bo \(\defun{f}{A}{B}\) funkcija. Tedaj je tip \(\isequiv(f)\) propozicija.
\end{izrek}

\begin{dokaz}
  Denimo, da je funkcija \(f\) ekvivalenca. Po prejšnji trditvi ima tedaj funkcija \(f\) kontraktibilen tip prerezov, simetrično pa velja tudi, da ima kontraktibilen tip retrakcij. Ker je produkt kontraktibilnih tipov kontraktibilen, je tedaj kontraktibilen tudi tip \(\isequiv(f)\). Sledi, da je \(\isequiv(f)\) propozicija.
\end{dokaz}

\subsection{Množice in krožnica \(\mathbb{S}^{1}\)}

Obrnimo se nazaj h tipu \(\isinv(f)\). Omenili smo, da je propozicionalnost (TODO reword) tega tipa lahko odvisna od strukture tipov \(A\) in \(B\). Fiksirajmo torej funkcijo \(f\) in denimo, da je tip \(\isinv(f)\) propozicija. Razmislili bomo, ali lahko iz tega izpeljemo kak pogoj na tipa \(A\) in \(B\).

Denimo najprej, da je funkcija \(f\) obrnljiva, torej, da obstaja funkcija \(\defun{g}{B}{A}\) s pripadajočima homotopijama \(\homotopy{H}{f \circ g}{\id{}}\) in \(\homotopy{K}{g \circ f}{\id{}}\). Tedaj lahko na sledeč način konstruiramo drugačno homotopijo tipa \(g \circ f \sim \id{}\):
\[gf \overset{gfK^{-1}}{\sim} gfgf \overset{gHf}{\sim} gf \overset{K}{\sim} \id{}.\]
Ker je po predpostavki tip \(\isinv(f)\) propozicija, sta elementa \((g, H, K)\)
in \((g, H, gfK^{-1} \cdot gHf \cdot K)\) identificirana. Brez škode za splošnost lahko poskrbimo, da sta identifikaciji na prvih dveh komponentah enaki \(\refl{}\), zato na tretji komponenti dobimo identifikacijo tipa \[gfK^{-1} \cdot gHf \cdot K = K.\]
Krajšamo lahko homotopijo \(K\) in nato funkcijo \(g\), tako da dobimo identifikacijo tipa \(Hf = fK\). Če povzamemo:

\begin{trditev}
  \label{inv-prop-coherence}
  Naj bo \(\defun{f}{A}{B}\) funkcija in denimo, da je njen tip inverzov propozicija. Tedaj za vsak inverz \((g, H, K)\) velja identifikacija \(Hf = fK\).
\end{trditev}


Ker smo v strukturi inverza predpostavili obstoj dveh homotopij \(H\) in \(K\), ki obe govorita o funkcijah \(f\) in \(g\), sta nastali dve različni možnosti za homotopijo tipa \(f \circ g \circ f \sim f\), \(Hf\) in \(fK\), če pa bi bil tip \(\isinv(f)\) propozicija, bi bili možnosti identificirani. Ali pa taka identifikacija nujno vedno obstaja? Na prvi pogled ni očitno, da bi morali biti homotopiji \(H\) in \(K\) v kakeršnemkoli smislu skladni, saj sta povsem poljubni.

Tu nastopi pogoj na tip \(B\), saj so homotopije med funkcijami s kodomeno \(B\) pravzaprav družine identifikacij v tipu \(B\). Če bi v tipu \(B\) med poljubnima točkama obstajala največ ena identifikacija, bi lahko pokazali, da tudi med poljubnima funkcijama s kodomeno \(B\) obstaja največ ena homotopija. Take tipe imenujemo \emph{množice}.

\begin{definicija}
  Naj bo \(A\) tip. Pravimo, da je tip \(A\) \emph{množica}, če obstaja element tipa
  \[\pitype{x, y}{A}{\isprop(x = y)}.\]
\end{definicija}

Dokažimo torej, da med funkcijama, katerih kodomena je množica, obstaja največ ena homotopija.

\begin{trditev}
  \label{is-prop-htpy-set}
  Naj bosta \(A\) in \(B\) tipa ter denimo, da je tip \(B\) množica. Tedaj velja
  \[\pitype{f, g}{A \to B}{\isprop(f \sim g)}.\]
\end{trditev}

\begin{dokaz}
  Naj bosta \(f\) in \(g\) funkciji in denimo, da sta \(H\) in \(K\) homotopiji tipa \(f \sim g\). Konstruirati želimo homotopijo med homotopijama \(H\) in \(K\), torej funkcijo tipa
  \(\pitype{x}{A}{H(x) = K(x)}\), ki bi jo z uporabo funkcijske ekstenzionalnosti spremenili v identifikacijo med \(H\) in \(K\). Naj bo torej \(\typejudgment{x}{A}\).
  Tako \(H(x)\) kot \(K(x)\) sta identifikaciji tipa \(f(x) = g(x)\), ker pa je tip \(B\) množica, je tip \(f(x) = g(x)\) propozicija. Sledi, da sta elementa \(H(x)\) in \(K(x)\) identificirana.
\end{dokaz}

Vrnimo se k prejšnji diskusiji. Če nadaljno predpostavimo, da je tip \(B\) množica, tedaj po prejšnji trditvi sledi, da je tip \(f \circ g \circ f \sim f\) propozicija, torej sta homotopiji \(Hf\) in \(fK\) identificirani. Izkaže se tudi, da je bila to edina omejitev za propozicionalnst tipa \(\isinv(f)\) in še več, za ekvivalenco med tipoma \(\isequiv(f)\) in \(\isinv(f)\). Dokaz tega dejstva je podobno tehničen kot dokaz, da imajo ekvivalence kontraktibilen tip prerezov, zato ga bomo izpustili.

Omenimo lahko še, da je bila izbira tipa \(B\) kot tistega, za katerega smo izpeljali pogoj, arbitrarna. To sloni na dejstvu, da tudi za lastnost množice velja, da jo ekvivalence ohranjajo. Denimo namreč, da je tip \(A\) množica. Za dokaz propozicionalnosti \(\isinv(f)\) bomo zopet uporabili alternativno formulacijo propozicij, zato denimo, da je tip funkcija \(f\) obrnljiva. Dokazati želimo, da je tedaj tip \(\isinv(f)\) kontraktibilen. Ker je funkcija \(f\) obrnljiva, sta po trditi \ref{inv-of-equiv} tipa \(A\) in \(B\) tudi ekvivalentna, od koder sledi, da je tip \(B\) množica. Po prejšnji diskusiji je tedaj tip \(\isinv(f)\) propozicija, ker pa je po predpostavki tudi naseljen, je kontraktibilen. Velja torej sledeča trditev:

\begin{izrek}
  Naj bo \(\defun{f}{A}{B}\) funkcija in denimo, da je vsaj eden izmed tipov \(A\) in \(B\) množica. Tedaj je tip \(\isinv(f)\) propozicija.
\end{izrek}

Da bi poiskali funkcijo \(\defun{f}{A}{B}\), katere tip inverzov ni propozicija, jo moramo torej poiskati med funkcijami, katerih domena in kodomena nista množici. Izkaže pa se, da je obstoj tipov, ki niso množice, od Martin-Löfove teorije tipov neodvisen. Edini konstruktor za konstrukcijo identifikacij, ki ga imamo, je namreč \(\refl{}\), kljub temu pa obstoja drugih identifikacij ne moramo ne dokazati, ne ovreči.

Teorijo tipov lahko zato razširimo na enega izmed dveh načinov: razširimo jo lahko z aksiomom UIP (ang. \emph{Uniqueness of Identity Proofs}), ki v grobem zatrdi, da za vsak tip \(A\) in vsak element \(\typejudgment{a}{A}\) velja, da je tip \(a = a\) kontraktibilen na element \(\refl{a}\). Od tod posledično sledi tudi, da je vsak tip množica. Alternativno lahko za razvoj sintetične homotopske teorije predpostavimo obstoj različnih tipov, imenovanih \emph{višji induktivni tipi}, ki poleg poleg običajnih konstruktorjev za elemente vsebujejo tudi konstruktorje za identifikacije. V tem delu bomo predstavili enega najpreprostejših primerov višjih induktivnih tipov, \emph{krožnico} \(\krog\), in pokazali, da za identitetno funkcijo \(\id{\krog}\) velja, da \(\isinv(\id{\krog})\) \emph{ni} propozicija.

\pagebreak
\begin{definicija}
  \emph{Krožnica} \(\krog\) je podana s sledečimi pravili sklepanja:
  \begin{enumerate}
  \item Krožnica ima točko \(\typejudgment{\base}{\krog}\).
  \item V krožnici velja identifikacija \(\typejudgment{\lop}{\base = \base}\).
  \item Za krožnico velja sledeči princip indukcije. Naj bo \(B\) družina tipov nad \(\krog\) in denimo, da imamo element \(\typejudgment{u}{B(\base)}\) ter identifikacijo \(\typejudgment{p}{\transp{B}(\lop, u) = u}\). Tedaj velja
    \[\typejudgment{\ind{\krog}(u, p)}{\pitype{x}{\krog}{B(x)}}.\]
  \item Velja sodbena enakost \(\ind{\krog}(u, p)(\base) \doteq u\).
  \item Velja določena sodbena enakost, ki vključuje identifikacijo \(p\) in aplikacijo funkcije \(\ind{\krog}(u, p)\) na identifikacijo \(\lop\).
  \end{enumerate}
\end{definicija}

Zadnje pravilo sklepanja smo izpustili, saj njegova formulacija vključuje aplikacijo \emph{odvisnih} funkcij na identifikacije, ki je v tem delu nismo vpeljali. Kakorkoli tega pravila sklepanja ne bomo potrebovali.

Če v tretjem pravilu sklepanja družino tipov \(B\) nadomestimo s tipom \(B\), neodvsnim od \(\krog\), se pravilo sklepanja poenostavi v trditev, da za konstrukcijo funkcije tipa \(\defun{f}{\krog}{B}\) zadošča, da podamo element \(\typejudgment{u}{B}\) in identifikacijo \(\eqtype{p}{u}{u}\). Zadošča torej, da smo v tipu \(B\) prepoznali instanco krožnice: bazno točko ter identifikacijo na njej. Zadnje pravilo sklepanja tedaj zatrdi, da velja sodbena enakost \(\ap{f} p \doteq \lop\).

Sledi kratek intermezzo o \emph{ošpičenih tipih} in \emph{ošpičenih funkcijah}.
Pravimo, da je tip \(A\) ošpičen, če je opremljen z odlikovanim elementom \(\typejudgment{a}{A}\). Ošpičene tipe pogosto označimo tudi kot pare \((A, a)\). Pravimo, da je funkcija med ošpičenima tipoma ošpičena, če ohranja njuna odlikovana elementa.
\begin{definicija}
  Naj bosta \((A, a)\) in \((B, b)\) ošpičena tipa. Definiramo tip \emph{ošpičenih funkcij}
  \[(A, a) \to_{\ast} (B, b) := \sumtype{f}{A \to B}{f(a) = b}.\]
  Kadar sta odlikovani točki jasni, pogosto pišemo tudi \(A \to_{\ast} B\).
\end{definicija}

Pojem smo vpeljali, da lahko formuliramo sledečo trditev:
\begin{trditev}
  \label{circle-loop-space}
  Naj bo \((A, a)\) ošpičen tip. Tedaj sta tipa \((\krog, \base) \to_{\ast} (A, a)\) in \(a = a\) ekvivalentna.
\end{trditev}

\begin{dokaz}
  Naj bo \(\depfun{f}{(\krog, \base)}{(A, a)}\) ošpičena funkcija. Tedaj je \(\ap{f} \lop\) identifikacija tipa \(f(\base) = f(\base)\), ker pa je funkcija ošpičena, imamo tudi identifikacijo tipa \(f(\base) = a\). Združimo ju lahko v identifikacijo tipa \(a = a\).

  Naj bo obratno \(\eqtype{p}{a}{a}\) identifikacija. Po indukcijskem principu krožnice je tedaj \(\ind{\krog}(a, p)\) funkcija tipa \(\krog \to A\), ker pa velja sodbena enakost \(\ind{\krog}(a, p)(\base) \doteq a\), je tudi primerno ošpičena.

  Dokaz vzajemne inverznosti predpisov bomo izpustili, sloni pa na zadnjem pravilu sklepanja za krožinco.
\end{dokaz}

Osredotočimo se sedaj na obrnljivost funkcije \(\defun{\id{\krog}}{\krog}{\krog}\). Za njen inverz lahko podamo kar funkcijo \(\id{\krog}\), podati pa moramo še dve homotopiji tipa \(\id{\krog} \circ \id{\krog} \sim \id{\krog}\), torej dve funkciji tipa \(\pitype{x}{\krog}{x = x}\). Prvo homotopijo, imenujmo jo \(H\), definiramo kot konstantno homotopijo \(\lambda x.\refl{x}\). V sledečem bomo konstruirali še homotopijo \(K\), ki v določenem smislu za vsako točko \(x\) poda identifikacijo med \(x\) in \(x\), ki enkrat zaokroži okoli krožnice. V posebnem velja \(K(\base) \doteq \lop\).

Če bi bil tip \(\isinv(\id{\krog})\) propozicija, bi bili homotopiji \(H\) in \(K\) tedaj skladni in veljala bi identifikacija \(H\id{\krog} = \id{\krog}K\), kar pa lahko nadaljno poenostavimo v identifikacijo tipa \(H = K\). Posledično bi veljalo \(H \sim K\), torej bi veljala tudi identifikacija
\[\refl{\base} \doteq H(\base) = K(\base) \doteq \lop,\]
kar pa je protislovje.

Preostane nam, da konstruiramo omenjeno homotopijo \(K\), za to pa najprej potrebujemo določeno lemo:

\begin{lema}
  \label{transp-lemma}
  Naj bo \(A\) tip in \(\typejudgment{x, y}{A}\). Denimo, da imamo identifikacije \(\eqtype{p}{x}{y}\), \(\eqtype{q}{x}{x}\) in \(\eqtype{r}{y}{y}\), za katere velja identifikacija  \(q \cdot p = p \cdot r\). Tedaj velja identifikacija \(\transp{\lambda a. a = a}(p, q) = r\).
\end{lema}

\begin{dokaz}
  Prepričajmo se najprej, da je lema smiselna. Transport vzdolž identifikacije \(p\) se dogaja znotraj družine tipov \(a = a\), indeksirane z elementi \(\typejudgment{a}{A}\). Identifikaciji \(q\) in \(r\) sta torej elementa tipov te družine pri elementih \(x\) in \(y\).

  Po principu eliminacije identifikacij zadošča pokazati, da lema velja za \(p \doteq \refl{x}\). Ker je \(\refl{}\) enota za konkatenacijo, se predpostavka poenostavi v identifikacijo \(q = r\), ker pa je transport vzdolž \(\refl{}\) identiteta, se v \(q = r\) poenostavi tudi željena identifikacija, kar zaključi dokaz.
\end{dokaz}

Sedaj lahko konstruiramo željeno homotopijo \(K\). Po indukcijskem principu za krožnico velja, da za konstrukcijo funkcije \(\pitype{x}{\krog}{x = x}\) zadošča, da podamo element \(\eqtype{u}{\base}{\base}\) ter identifikacijo \(\transp{\lambda x. x = x}(\lop, u) = u\). Veljala bo tudi sodbena enakost \(K(\base) \doteq u\).
Podamo lahko \(\eqtype{\lop}{\base}{\base}\), ker pa očitno velja identifikacija \(\lop \cdot \lop = \lop \cdot \lop\), tedaj po prejšnji lemi sledi, da velja tudi željena identifikacija \(\transp{\lambda x. x = x}(\lop, \lop) = \lop.\)
Kot željeno velja \(K(\base) = \lop\). Diskusijo povzemimo v sledeči trditvi.

\begin{trditev}
  \label{inv-circle-neg-prop}
  Tip inverzov funkcije \(\defun{\id{\krog}}{\krog}{\krog}\) ni propozicija.
\end{trditev}

\begin{dokaz}
  Denimo, da je tip inverzov funkcije \(\id{\krog}\) propozicija. Ker imamo identifikacijo \(\eqtype{\refl{}}{\lop \cdot \lop}{\lop \cdot \lop}\), imamo po lemi \ref{transp-lemma} tudi identifikacijo \(\eqtype{p}{\transp{}(\lop, \lop)}{\lop}\). Trojica
  \((\id{\krog}, \lambda x. \refl{x}, \ind{\krog}(\lop, p)\)) je tedaj inverz funkcije \(\id{\krog}\), ker pa je \(\isinv(\id{\krog})\) propozicija, po trditvi \ref{inv-prop-coherence} sledi, da imamo identifikacijo med homotopijama
  \(\lambda x.\refl{x}\) in \(\ind{\krog}(\lop, p)\). Če obe homotopiji evaluiramo pri elementu \(\base\), dobimo identifikacijo med identifikacija \(\refl{\base}\) in \(\lop\), kar pa je protislovje.
\end{dokaz}

Za konec tega poglavja imamo sledeči izrek:

\begin{izrek}
  Tipa \(\isequiv(f)\) in \(\isinv(f)\) nista ekvivalentna za vsako funkcijo \(f\).
\end{izrek}

\begin{dokaz}
  Po izreku \ref{is-equiv-prop} je tip \(\isequiv(\id{\krog})\) propozicija, tip \(\isinv(\id\krog)\) pa po trditvi \ref{inv-circle-neg-prop} ni. Ker po trditvi \ref{equiv-preserves-prop} ekvivalence ohranjajo lastnost propozicij, tipa nista ekvivalentna.
\end{dokaz}

Še enkrat lahko poudarimo, da razlika med pojmoma ekvivalence in obrnljivosti leži izključno v višji strukturi identifikacij in homotopij. V strukturi ekvivalence sta homotopiji prereza in retrakcije druga od druge ločeni, zato tip ostane propozicija ne glede na strukturo tipov \(A\) in \(B\). Kot pa smo videli, sta homotopiji v strukturi inverza med seboj prepleteni, zato se tip inverzov lahko navzame višje strukture identifikacij, ki je prisotna v tipih \(A\) in \(B\).

V tem poglavju smo na protiprimeru \emph{ene} funkcije predstavili razliko med pojmom obrnljivosti in pojmom ekvivalence, v sledečem poglavju pa se bomo bolj kvalitativno poglobili v povezavo med tipom \emph{vseh} obrnljivih funkcij in tipom \emph{vseh} ekvivalenc med tipoma \(A\) in \(B\).

%%% Local Variables:
%%% mode: latex
%%% TeX-master: "../Najdovski-27191110-2023"
%%% End:
