\section{Sintetična homotopska teorija}


\subsection{Identifikacije in homotopije}
Osnovna motivacija za vpeljavo tipa identifikacij je dejstvo, da je pojem sodbene enakosti
\emph{prestrog} in da prek nje pogosto ne moremo identificirati vseh izrazov, ki bi
jih želeli. Poleg tega si v skladu z interpretacijo \emph{izjave kot tipi} želimo tip, elementi
katerega bi predstavljali dokaze enakosti med dvema elementoma.

Pomankljivost sodbene enakosti si lahko ogledamo na primeru
komutativnosti naravnih števil.

\begin{primer}
  Seštevanje naravnih števil bomo definirali kot funkcijo
  \[\defun{\typename{sum}}{\N}{(\N \to \N)},\] njeno evaluacijo \(\typename{sum}(m, n)\)
  pa bomo kot običajno pisali kot \(m + n\).
  Ker je tip \(\N \to (\N \to \N)\) po opombi \ref{neodvisna-vsota} sodbeno enak tipu
  \(\pitype{n}{\N}{\N \to \N}\), lahko njegove elemente konstruiramo po indukcijskem
  principu naravnih števil.
  Izpuščajoč nekaj podrobnosti to pomeni, da želimo vrednost izraza \(m + n\) definirati z
  indukcijo na \(n\), torej moramo v baznem primeru podati vrednost izraza \(m + \natzero\),
  v primeru koraka pa moramo pod predpostavko, da poznamo vrednost izraza \(m + n\),
  podati vrednost izraza \(m + \natsucc(n)\). Naravno je,
  da izberemo \(m + \natzero :\doteq m\) in \(m + \natsucc(n) :\doteq\natsucc(m + n)\).
  Za poljubni \emph{fiksni} naravni števili, recimo
  \(\typename{4}\) in \(\typename{5}\), lahko sedaj z uporabo pravil sklepanja za naravna
  števila pod točko \ref{naturals} dokažemo, da velja
  \(\typename{4} + \typename{5} \doteq \typename{5} + \typename{4}\). Če pa sta po drugi strani
  \(m\) in \(n\)
  \emph{spremenljivki} tipa \(\N\), tedaj zaradi asimetričnosti definicije seštevanja
  enakosti \(n + m \doteq m + n\) ne moremo
  dokazati. Potrebujemo tip \emph{identifikacij} \(\Id{\N}\), z uporabo katerega bi
  lahko z indukcijo pokazali, da velja \[\pitype{m, n}{\N}{\Id{\N}(n + m, m + n)}.\]
\end{primer}

\begin{definicija}
  Naj bo \(A\) tip in \(\typejudgment{a, x}{A}\). Tedaj lahko tvorimo tip
  \(\Id{A}(a, x)\),
  imenovan \emph{tip identifikacij med elementoma a in x}. Zanj veljajo sledeča pravila
  sklepanja:
  \begin{enumerate}
  \item Za vsak element \(\typejudgment{a}{A}\) obstaja identifikacija
    \(\typejudgment{\refl{a}}{\Id{A}(a, a)}\).
  \item Velja sledeč princip \emph{eliminacije identifikacij}.
    Naj bo \(\typejudgment{a}{A}\), \(B\) družina tipov, indeksirana z
    \(\typejudgment{x}{A}\) ter \(\typejudgment{p}{\Id{A}(a, x)}\) in denimo, da velja
    \(\typejudgment{r(a)}{B(a, \refl{a})}\). Tedaj velja
    \[\typejudgment{\typename{ind}_{\Id{A}}(a, r(a))}
      {\pitype{x}{A}{\pitype{p}{\Id{A}(a, x)}{B(x, p)}}}\]
  \item Za vsak \(\typejudgment{a}{A}\) velja enakost
    \(\typename{ind}_{\Id{A}}(a, r(a))(a, \refl{a}) \doteq r(a)\).
  \end{enumerate}
  Tip identifikacij \(\Id{A}(a, x)\) bomo od sedaj pisali kar kot \(a = x\) in za boljšo
  berljivost izpuščali ekspliciten zapis tipa \(A\). Kadar je \(p\) identifikacija
  med \(a\) in \(x\), bomo torej pisali \(\eqtype{p}{a}{x}\).

  Konstruktor \(\eqtype{\refl{a}}{a}{a}\)
  nam zagotavlja, da za vsak element \(\typejudgment{a}{A}\) obstaja kanonična
  identifikacija elementa samega s sabo. Princip eliminacije identifikacij nato zatrdi,
  da če želimo za vsako identifikacijo \(\eqtype{p}{a}{x}\) konstruirati element tipa
  \(B(x, p)\), tedaj zadošča, da konstruiramo le element tipa \(B(a, \refl{a})\).
  Pravimo tudi, da je družina tipov \(a = x\) \emph{induktivno definirana} s
  konstruktorjem \(\refl{a}\).
.\end{definicija}

  Resnična motivacija za takšno definicijo tipa identifikacij sega pregloboko, da bi se
  vanjo spustili v tem delu, povemo pa lahko, da lahko preko nje dokažemo vse željene
  lastnosti, ki bi jih od identifikacij pričakovali. Nekaj od teh vključuje:
\begin{enumerate}
\item Velja \emph{simetričnost}: za vsaka elementa \(\typejudgment{x, y}{A}\) obstaja
  funkcija \[\defun{\typename{sym}}{(x = y)}{(y = x)}.\] Identifikacijo \(\typename{sym}(p)\)
  imenujemo \emph{inverz} identifikacije \(p\) in ga pogosto označimo z \(p^{-1}\).
\item Velja \emph{tranzitivnost}: za vsake elemente \(\typejudgment{x, y, z}{A}\) obstaja
  funkcija \[\defun{\typename{concat}}{(x = y)}{((y = z) \to (x = z))},\] ki jo imenujemo
  \emph{konkatenacija identifikacij}, \(\typename{concat}(p, q)\) pa pogosto označimo
  z \(p \cdot q\).
\item Funkcije ohranjajo identifikacije: za vsaka elementa \(\typejudgment{x, y}{A}\)
  in vsako funkcijo \(\defun{f}{A}{B}\) obstaja funkcija
  \[\defun{\typename{ap}_{f}}{(x = y)}{(f(x) = f(y))},\]
  ki jo imenujemo \emph{aplikacija} funkcije \(f\) na identifikacije v \(A\).
\item Družine tipov spoštujejo identifikacije: za vsaka elementa
  \(\typejudgment{x, y}{A}\), identifikacijo \(\eqtype{p}{x}{y}\) in družino tipov
  \(B\) nad \(A\) obstaja funkcija
  \[\defun{\typename{tr}_{B}(p)}{B(x)}{B(y)}.\] To pomeni, da lahko elemente tipa
  \(B(x)\) \emph{transportiramo} vzdolž identifikacije \(\eqtype{p}{x}{y}\), da dobimo
  elemente \(B(y)\).
  V interpretaciji \emph{izjave kot tipi} to pomeni, da če je \(P\) predikat na \(A\)
  in velja identifikacija \(x = y\), da tedaj \(P(x)\) velja natanko tedaj kot \(P(y)\),
  saj lahko transport uporabimo tudi na \(p^{-1}\).
\end{enumerate}
Konstrukcije naštetih funkcij lahko poiščemo v 5.~poglavju (TODO cite intro to hott), vse
izmed njih pa zahtevajo le preprosto uporabo eliminacije identifikacij.
(TODO grupoidala struktura tipov, interpretacija identifikacij kot poti)

Izkaže se, da smo s tako definicijo identifikacij zelo omejeni v konstruiranju identifikacij
med funkcijami. Namesto tega lahko identifikacije med funkcijami nadomestimo s
t.i.~\emph{homotopijami}, ki v mnogih primerih zadoščajo, konstruiramo pa jih veliko lažje.
Homotopija med funkcijami zatrdi, da se ti ujemata na vsakem elementu domene.
\begin{definicija}
  Naj bosta \(A\) in \(B\) tipa ter \(f\) in \(g\) funkciji tipa \(A \to B\). Definiramo tip
  \[f \sim g := \pitype{x}{A}{f(x) = g(x)},\]
  imenovan \emph{tip homotopij med f in g}, njegove elemente pa imenujemo \emph{homotopije}.
\end{definicija}

\begin{opomba}
  Na kratko komentiramo o podobnosti med homotopijami v teoriji tipov in homotopijami v
  klasični topologiji. Naj bosta \(X\) in \(Y\) topološka prostora in
  \(\defun{f, g}{X}{Y}\) zvezni
  funkciji. Z \(I\) označimo interval \([0, 1]\). Homotopijo med funkcijama \(f\) in \(g\)
  tedaj definiramo kot zvezno funkcijo \(\defun{H}{I \times X}{Y}\), za katero velja
  \(H(0, x) = f(x)\) in \(H(1, x) = g(x)\). Ker pa je \(I\) lokalno kompakten prostor, so
  zvezne funkcije \(\defun{H}{I \times X}{Y}\) v bijektivni korespondenci z zveznimi funkcijami
  \(\defun{\hat{H}}{X}{\mathcal{C}(I, Y)}\), kjer je \(\mathcal{C}(I, Y)\) prostor zveznih funkcij med \(I\) in
  \(Y\), opremljen s kompaktno-odprto topologijo. Zvezne funkcije \(\defun{\alpha}{I}{Y}\)
  ustrezajo potem v \(Y\) med točkama \(\alpha(0)\) in \(\alpha(1)\). Če je torej \(H\)
  homotopija med \(f\) in \(g\), za pripadajočo funkcijo \(\hat{H}\) velja, da je
  \(\hat{H}(x)\) pot med točkama \(f(x)\) in \(g(x)\) za vsak \(x \in X\). To je analogno
  definiciji homotopij med funkcijama \(f\) in \(g\) v teoriji tipov, ki vsaki točki
  \(\typejudgment{x}{X}\) priredijo identifikacijo \(f(x) = g(x)\).
\end{opomba}

Operacije na identifikacijah lahko po elementih razširimo do operacij na homotopijah,
za delo s homotopijami
pa bomo potrebovali še operaciji (TODO whiskering?), s katerima lahko homotopije z leve ali
z desne razširimo s funkcijo.

\begin{trditev}
  Naj bosta \(A\) in \(B\) tipa in \(\defun{f, g, h}{A}{B}\) funkcije. Denimo, da obstajata
  homotopiji \(\homotopy{H}{f}{g}\) in \(\homotopy{K}{g}{h}\).

  Tedaj obstajata homotopiji \(\homotopy{H^{-1}}{g}{f}\) in \(\homotopy{H \cdot K}{f}{h}\).
\end{trditev}

\begin{dokaz}
  Če homotopiji H in K evaluiramo na elementu \(\typejudgment{x}{A}\), dobimo identifikaciji
  \(\eqtype{H(x)}{f(x)}{g(x)}\) in \(\eqtype{K(x)}{g(x)}{h(x)}\). Homotopijo \(H^{-1}\) lahko
  torej definiramo kot \(\lambda x. H(x)^{-1}\), homotopijo \(H \cdot K\) pa kot \(\lambda x.(H(x) \cdot K(x))\).
\end{dokaz}

\begin{trditev}
  Naj bodo \(A, B, C\) in \(D\) tipi ter \(\defun{h}{A}{B}\), \(\defun{f, g}{B}{C}\) in
  \(\defun{k}{C}{D}\) funkcije. Denimo, da obstaja homotopija \(\homotopy{H}{f}{g}\).
  Tedaj obstajata homotopiji \(\homotopy{Hh}{f \circ h}{g \circ h}\) in
  \(\homotopy{kH}{k \circ f}{k \circ g}\).
\end{trditev}

\begin{dokaz}
  Homotopijo \(H\) zopet evaluiramo na elementu \(\typejudgment{y}{B}\) in dobimo
  identifikacijo \(\eqtype{H(y)}{f(y)}{g(y)}\). Če sedaj nanjo apliciramo funkcijo \(h\),
  dobimo identifikacijo \(\eqtype{\ap{h}(H(y))}{h(f(y))}{h(g(y))}\). Homotopijo \(kH\) lahko
  torej definiramo kot
  \(\lambda y. \typename{ap}_{k}(H(y))\).

  Naj bo sedaj \(\typejudgment{x}{A}\). Homotopijo \(H\) lahko tedaj evaluiramo tudi
  na elementu \(h(x)\) in tako dobimo identifikacijo
  \(\eqtype{H(h(x))}{f(h(x))}{g(h(x))}\). Homotopijo \(Hh\) lahko torej definiramo kot
  \(\lambda x.H(h(x))\).
\end{dokaz}

\subsection{Obrnljivost in ekvivalenca}


\begin{definicija}
  Za funkcijo \(f\) pravimo, da je \emph{obrnljiva} oziroma, da \emph{ima inverz},
  če obstaja element tipa
  \[\isinv(f) := \sumtype{g}{B \to A}{(f \circ g \sim id) \times (g \circ f \sim id)},\]
  imenovanega \emph{tip inverzov funkcije f}.

  Funkcije, pripadajoče elementom \(\isinv(f)\), imenujemo \emph{inverzi funkcije f}.
\end{definicija}

\begin{definicija}
  Za funkcijo \(f\) pravimo, da \emph{ima prerez}, če obstaja element tipa
  \[\sect(f) := \sumtype{g}{B \to A}{f \circ g \sim id},\]
  imenovanega \emph{tip prerezov funkcije f}.

  Funkcije, pripadajoče elementom tipa \(\sect f\) imenujemo \emph{prerezi funkcije f}.
  Za funkcijo \(f\) pravimo, da \emph{ima retrakcijo}, če obstaja element tipa
  \[\ret(f) := \sumtype{g}{B \to A}{g \circ f \sim id},\]
  imenovanega \emph{tip retrakcij funkcije f}.

  Funkcije, pripadajoče elementom tipa \(\ret f\) imenujemo \emph{retrakcije funkcije f}.
\end{definicija}

\begin{definicija}
  Pravimo, da je funkcija \(f\) \emph{ekvivalenca}, če ima tako prerez kot retrakcijo,
  torej, če obstaja element tipa \[\isequiv(f) := \sect(f) \times \ret(f).\]
  Pravimo, da je tip \(A\) \emph{ekvivalenten} tipu \(B\), če obstaja ekvivalenca med
  njima, torej element tipa \(A \simeq B := \sumtype{f}{A \to B}{\isequiv(f)}\). Funkcijo,
  pripadajočo elementu \(\dequiv{e}{A}{B}\), označimo z \(\map e\).
\end{definicija}

\subsection{Osnovne lastnosti}

V definiciji obrnljivosti smo zahtevali, da ima funkcija obojestranski inverz, v
definiciji ekvivalence pa smo zahtevali le, da ima ločen levi in desni inverz.
To bi nas lahko
napeljalo k prepričanju, da je pojem obrnljivosti močnejši od pojma ekvivalence, vendar
spodnja trditev pokaže da sta pravzaprav logično ekvivalentna. Pokazali bomo, da lahko
prerez (ali simetrično, retrakcijo) ekvivalence \(f\) vedno izboljšamo do inverza,
kar pokaže, da je \(f\) tudi obrnljiva.

\begin{trditev}
  \label{inv-of-equiv}
  Funkcija je ekvivalenca natanko tedaj, ko je obrnljiva.
\end{trditev}

\begin{dokaz}
  Denimo, da je funkcija \(f\) obrnljiva. Tedaj lahko njen inverz podamo tako kot njen
  prerez, kot njeno retrakcijo, kar pokaže, da je ekvivalenca.

  Obratno denimo, da je funkcija \(f\) ekvivalenca. Podan imamo njen prerez \(s\) s
  homotopijo \(\homotopy{H}{f \circ s}{id}\) in njeno retrakcijo \(r\) s homotopijo
  \(\homotopy{K}{r \circ f}{id}\), s katerimi lahko konstruiramo homotopijo tipa
  \(s \circ f \sim id\) po sledečem izračunu:
  \[s f \overset{K^{-1}sf}{\sim} r f s f \overset{rHf}{\sim} r f \overset{K}{\sim} id. \qedhere\]
\end{dokaz}

\begin{definicija}
  Naj bo \(\typejudgment{f}{\pitype{x}{A}{\left(Bx \to Cx\right)}}\) družina funkcij.
  Definiramo funkcijo
  \(\typejudgment{\tot(f)}{\sumtype{x}{A}{Bx}} \to \sumtype{x}{A}{Cx}\) s predpisom
  \(\tot(f)(x, y) = (x, f(x)(y))\).
\end{definicija}

\begin{trditev}
  Naj bo \(\typejudgment{f}{\pitype{x}{A}{\left(Bx \to Cx\right)}}\) družina funkcij
  in denimo, da je \(f(x)\) ekvivalenca za vsak \(\typejudgment{x}{A}\). Tedaj je
  ekvivalenca tudi funkcija \(\tot(f)\).
\end{trditev}

\begin{dokaz}
  Ker je funkcija \(f(x)\) ekvivalenca za vsak \(\typejudgment{x}{A}\),
  lahko tvorimo družino funkcij
  \(\typejudgment{s}{\pitype{x}{A}{\left(Cx \to Bx\right)}}\), kjer je \(s(x)\)
  prerez funkcije \(f(x)\).
  Trdimo, da je tedaj \(\tot(s)\) prerez funkcije \(\tot(f)\), saj
  za vsak \(\typejudgment{(x, y)}{\sumtype{x}{A}{Cx}}\) velja enakost
  \[\tot(f)(\tot(s)(x, y)) = \tot(f)(x, s(x)(y)) = (x, f(x)(s(x)(y))) = (x, y).\]
  Analogno lahko konstruiramo retrakcijo funkcije \(\tot(f)\), kar zaključi dokaz.
\end{dokaz}

Zgornja trditev je pomembna, saj nam omogoči, da konstrukcijo ekvivalence med odvisnima
vsotama z istim baznim tipom poenostavimo na konstrukcijo družine ekvivalenc,
kar je pogosto veliko lažje. Uporabljali ga bomo v obliki sledeče posledice:

\begin{posledica}
  \label{equiv-tot}
  Naj bo \(A\) tip, \(B\) in \(C\) družini tipov nad \(A\) in denimo, da velja
  \(Bx \simeq Cx\) za vsak \(\typejudgment{x}{A}\).
  Tedaj velja \(\sumtype{x}{A}{Bx} \simeq \sumtype{x}{A}{Cx}\).
\end{posledica}

%%% Local Variables:
%%% mode: latex
%%% TeX-master: "../Najdovski-27191110-2023"
%%% End:
