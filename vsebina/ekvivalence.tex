\section{Ekvivalence in obrnljive funkcije}

\begin{definicija}
  Naj bo \(\defun{f}{A}{B}\) funkcija. Tip
  \[\sect f := \sumtype{g}{B \to A}{f \circ g \sim id}\]
  imenujemo \emph{tip prerezov funkcije f}. Za funkcijo \(f\) pravimo, da
  \emph{ima prerez}, če obstaja element tipa \(\sect(f)\), imenovan \emph{prerez f}.

  Tip \[\ret(f) := \sumtype{g}{B \to A}{g \circ f \sim id}\] imenujemo
  \emph{tip retrakcij funkcije f}. Za funkcijo \(f\) pravimo, da \emph{ima retrakcijo}, če
  obstaja element tipa \(\ret(f)\), imenovan \emph{retrakcija f}.
\end{definicija}

\begin{definicija}
  Pravimo, da je funkcija \(f\) \emph{ekvivalenca}, če ima tako prerez kot retrakcijo,
  torej, če obstaja element tipa \[\isequiv(f) := \sect(f) \times \ret(f).\]
  Pravimo, da je tip \(A\) \emph{ekvivalenten} tipu \(B\), če obstaja ekvivalenca med
  njima, torej element tipa \(A \simeq B := \sumtype{f}{A \to B}{\isequiv(f)}\).
\end{definicija}

\begin{definicija}
  Naj bo \(\defun{f}{A}{B}\) funkcija. Tip
  \[\isinv(f) := \sumtype{g}{B \to A}{(f \circ g \sim id) \times (g \circ f \sim id)}\]
  imenujemo \emph{tip inverzov funkcije f}. Za funkcijo \(f\) pravimo, da je
  \emph{obrnljiva} oziroma, da \emph{ima inverz}, če obstaja element tipa
  \(\isinv(f)\), imenovan \emph{inverz f}.
\end{definicija}

V definiciji obrnljivosti smo zahtevali, da ima funkcija obojestranski inverz, v
definiciji ekvivalence pa smo zahtevali le, da ima ločen levi in desni inverz.
To bi nas lahko
napeljalo k prepričanju, da je pojem obrnljivosti močnejši od pojma ekvivalence, vendar
spodnja trditev pokaže da sta pravzaprav logično ekvivalentna. Pokazali bomo, da lahko
prerez (ali simetrično, retrakcijo) ekvivalence \(f\) vedno izboljšamo do inverza,
kar pokaže, da je \(f\) tudi obrnljiva.

\begin{trditev}
  Funkcija je ekvivalenca natanko tedaj, ko je obrnljiva.
\end{trditev}

\begin{dokaz}
  Denimo, da je funkcija \(f\) obrnljiva. Tedaj lahko njen inverz podamo tako kot njen
  prerez, kot njeno retrakcijo, kar pokaže, da je ekvivalenca.

  Obratno denimo, da je funkcija \(f\) ekvivalenca. Podano imamo funkcijo \(s\) s
  homotopijo \(\homotopy{H}{f \circ s}{id}\) in funkcijo \(r\) s homotopijo
  \(\homotopy{K}{r \circ f}{id}\), s katerimi lahko konstruiramo homotopijo tipa
  \(s \circ f \sim id\) po sledečem izračunu:
  \[s f \overset{K^{-1}sf}{\sim} r f s f \overset{rHf}{\sim} r f \overset{K}{\sim} id. \qedhere\]
\end{dokaz}

\begin{comment}
\begin{definicija}
  Naj bo \(\defun{f}{A}{B}\) funkcija in \(\typejudgment{b}{B}\). Tip
  \[\fib{f}{b} := \sumtype{x}{A}{f(x) = b}\]
  imenujemo \emph{vlakno funkcije f pri točki b}.
\end{definicija}

\begin{trditev}
  Funkcija \(f\) je obrnljiva natanko tedaj, ko je vlakno funkcije \(f\) pri \(y\)
  kontraktibilno za vsak \(\typejudgment{y}{B}\).
\end{trditev}

\begin{dokaz}
  Denimo najprej, da je funkcija \(\defun{f}{A}{B}\) obrnljiva in naj bo
  \(\typejudgment{y}{B}\). Dokazati želimo, da je tip
  \(\fib{f}{y} = \sumtype{x}{A}{fx = y}\)
  kontraktibilen. Ker je funkcija \(f\) obrnljiva, obstaja funkcija \(\defun{g}{B}{A}\)
  in homotopija \(\homotopy{H}{f \circ g}{id}\), zato lahko za center kontrakcije
  tipa \(\fib{f}{y}\) izberemo element \(\left(g(y), H(y)\right)\). Da bi konstruirali
\end{dokaz}
\end{comment}

\begin{definicija}
  Naj bo \(\typejudgment{f}{\pitype{x}{A}{\left(B(x) \to C(x)\right)}}\) družina funkcij.
  Definiramo (TODO prevedi total) funkcijo
  \(\typejudgment{\tot(f)}{\sumtype{x}{A}{B(x)}} \to \sumtype{x}{A}{C(x)}\) s predpisom
  \(\tot(f)(x, y) = (x, f(x, y))\).
\end{definicija}

\begin{izrek}
  Naj bo \(\typejudgment{f}{\pitype{x}{A}{\left(B(x) \to C(x)\right)}}\) družina funkcij
  in denimo, da je \(f(x)\) ekvivalenca za vsak \(\typejudgment{x}{A}\). Tedaj je
  ekvivalenca tudi funkcija \(\tot(f)\).
\end{izrek}

\begin{dokaz}
  Ker je funkcija \(f(x)\) ekvivalenca za vsak \(\typejudgment{x}{A}\),
  lahko tvorimo družino funkcij
  \(\typejudgment{s}{\pitype{x}{A}{\left(C(x) \to B(x)\right)}}\), kjer je \(s(x)\)
  funkcija, pripadajoča prerezu funkcije \(f(x)\).
  Trdimo,
\end{dokaz}

Zgornji izrek je pomemben, saj nam omogoči, da konstrukcijo ekvivalenc med sigma tipi
nad istim baznim tipom poenostavimo na konstrukcijo družine ekvivalenc, kar je pogosto
veliko lažje. Uporabljali ga bomo v obliki sledeče posledice:

\begin{posledica}
  \label{equiv-tot}
  Naj bo \(A\) tip, \(B\) in \(C\) družini tipov nad \(A\) in denimo, da velja
  \(B(x) \simeq C(x)\) za vsak \(\typejudgment{x}{A}\).
  Tedaj velja \(\sumtype{x}{A}{B(x)} \simeq \sumtype{x}{A}{C(x)}\).
\end{posledica}

%%% Local Variables:
%%% mode: latex
%%% TeX-master: "../Najdovski-27191110-2023"
%%% End:
