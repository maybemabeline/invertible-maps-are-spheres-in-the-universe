\section{Ekvivalence in obrnljive funkcije}
\subsection{Definicije}
\begin{definicija}
  Za funkcijo \(f\) pravimo, da \emph{ima prerez}, če obstaja element tipa
  \[\sect f := \sumtype{g}{B \to A}{f \circ g \sim id},\]
  imenovanega \emph{tip prerezov funkcije f}.

  Funkcije, pripadajoče elementom tipa \(\sect f\) imenujemo \emph{prerezi funkcije f}.
  Za funkcijo \(f\) pravimo, da \emph{ima retrakcijo}, če obstaja element tipa
  \[\ret(f) := \sumtype{g}{B \to A}{g \circ f \sim id},\]
  imenovanega \emph{tip retrakcij funkcije f}.

  Funkcije, pripadajoče elementom tipa \(\ret f\) imenujemo \emph{retrakcije funkcije f}.
\end{definicija}

\begin{definicija}
  Pravimo, da je funkcija \(f\) \emph{ekvivalenca}, če ima tako prerez kot retrakcijo,
  torej, če obstaja element tipa \[\isequiv(f) := \sect(f) \times \ret(f).\]
  Pravimo, da je tip \(A\) \emph{ekvivalenten} tipu \(B\), če obstaja ekvivalenca med
  njima, torej element tipa \(A \simeq B := \sumtype{f}{A \to B}{\isequiv(f)}\). Funkcijo,
  pripadajočo elementu \(\dequiv{e}{A}{B}\), označimo z \(\map e\).
\end{definicija}

\begin{definicija}
  Za funkcijo \(f\) pravimo, da je \emph{obrnljiva} oziroma, da \emph{ima inverz},
  če obstaja element tipa
  \[\isinv(f) := \sumtype{g}{B \to A}{(f \circ g \sim id) \times (g \circ f \sim id)},\]
  imenovanega \emph{tip inverzov funkcije f}.

  Funkcije, pripadajoče elementom \(\isinv(f)\), imenujemo \emph{inverzi funkcije f}.
\end{definicija}

\subsection{Osnovne lastnosti}

V definiciji obrnljivosti smo zahtevali, da ima funkcija obojestranski inverz, v
definiciji ekvivalence pa smo zahtevali le, da ima ločen levi in desni inverz.
To bi nas lahko
napeljalo k prepričanju, da je pojem obrnljivosti močnejši od pojma ekvivalence, vendar
spodnja trditev pokaže da sta pravzaprav logično ekvivalentna. Pokazali bomo, da lahko
prerez (ali simetrično, retrakcijo) ekvivalence \(f\) vedno izboljšamo do inverza,
kar pokaže, da je \(f\) tudi obrnljiva.

\begin{trditev}
  \label{inv-of-equiv}
  Funkcija je ekvivalenca natanko tedaj, ko je obrnljiva.
\end{trditev}

\begin{dokaz}
  Denimo, da je funkcija \(f\) obrnljiva. Tedaj lahko njen inverz podamo tako kot njen
  prerez, kot njeno retrakcijo, kar pokaže, da je ekvivalenca.

  Obratno denimo, da je funkcija \(f\) ekvivalenca. Podan imamo njen prerez \(s\) s
  homotopijo \(\homotopy{H}{f \circ s}{id}\) in njeno retrakcijo \(r\) s homotopijo
  \(\homotopy{K}{r \circ f}{id}\), s katerimi lahko konstruiramo homotopijo tipa
  \(s \circ f \sim id\) po sledečem izračunu:
  \[s f \overset{K^{-1}sf}{\sim} r f s f \overset{rHf}{\sim} r f \overset{K}{\sim} id. \qedhere\]
\end{dokaz}

\begin{definicija}
  Naj bo \(\typejudgment{f}{\pitype{x}{A}{\left(Bx \to Cx\right)}}\) družina funkcij.
  Definiramo funkcijo
  \(\typejudgment{\tot(f)}{\sumtype{x}{A}{Bx}} \to \sumtype{x}{A}{Cx}\) s predpisom
  \(\tot(f)(x, y) = (x, f(x, y))\).
\end{definicija}

\begin{izrek}
  Naj bo \(\typejudgment{f}{\pitype{x}{A}{\left(Bx \to Cx\right)}}\) družina funkcij
  in denimo, da je \(f(x)\) ekvivalenca za vsak \(\typejudgment{x}{A}\). Tedaj je
  ekvivalenca tudi funkcija \(\tot(f)\).
\end{izrek}

\begin{dokaz}
  Ker je funkcija \(f(x)\) ekvivalenca za vsak \(\typejudgment{x}{A}\),
  lahko tvorimo družino funkcij
  \(\typejudgment{s}{\pitype{x}{A}{\left(Cx \to Bx\right)}}\), kjer je \(s(x)\)
  prerez funkcije \(f(x)\).
  Trdimo, da je tedaj \(\tot(s)\) prerez funkcije \(\tot(f)\), saj
  za vsak \(\typejudgment{(x, y)}{\sumtype{x}{A}{Cx}}\) velja enakost
  \[\tot(f)(\tot(s)(x, y)) = \tot(f)(x, s(x, y)) = (x, f(x, s(x, y))) = (x, y).\]
  Analogno lahko konstruiramo retrakcijo funkcije \(\tot(f)\), kar zaključi dokaz.
\end{dokaz}

Zgornji izrek je pomemben, saj nam omogoči, da konstrukcijo ekvivalenc med sigma tipi
nad istim baznim tipom poenostavimo na konstrukcijo družine ekvivalenc, kar je pogosto
veliko lažje. Uporabljali ga bomo v obliki sledeče posledice:

\begin{posledica}
  \label{equiv-tot}
  Naj bo \(A\) tip, \(B\) in \(C\) družini tipov nad \(A\) in denimo, da velja
  \(Bx \simeq Cx\) za vsak \(\typejudgment{x}{A}\).
  Tedaj velja \(\sumtype{x}{A}{Bx} \simeq \sumtype{x}{A}{Cx}\).
\end{posledica}

%%% Local Variables:
%%% mode: latex
%%% TeX-master: "../Najdovski-27191110-2023"
%%% End:
