\section{Osnove Martin-Löfove odvisne teorije tipov}

Sledeč (TODO cite intro to hott) bomo v tem poglavju predstavili osnovne koncepte
Martin-Löfove odvisne teorije tipov. Zaradi velike razsežnosti in dokajšnje tehničnosti
formalne predstavitve teorije tipov se ne bomo do potankosti spustili v njene podrobnosti,
vendar bomo koncepte predstavili v nekoliko neformalnem slogu, osredotočajoč se na
analogije z bolj standardnimi matematičnimi koncepti iz formalizma teorije množic.

\subsection{Sodbe in pravila sklepanja}

Osnovni gradniki teorije tipov so \emph{tipi}, ki jih bomo označevali z velikimi tiskanimi
črkami, kot so \(A, B\) in \(C\), ter njihovi \emph{elementi}, ki jih bomo označevali
z malimi tiskanimi črkami, kot so \(x, y, z\) in \(a, b, c\). Vsak element \(x\) ima
natanko določen tip \(A\), kar izrazimo s tako imenovano \emph{sodbo}
\(\typejudgment{x}{A}\), ki jo preberemo kot ``\(x\) je tipa \(A\)''.
Ta je v mnogih pogledih podobna relaciji \(x \in A\) iz teorije množic, vendar se od
nje razlikuje v nekaj pomembnih pogledih.

Prvič, v teoriji množic je relacija \(\in\) v določenem smislu \emph{globalna};
vsak matematični objekt je kodiran kot določena množica in za poljubni množici \(A\) in
\(B\) se lahko vprašamo, ali velja \(A \in B\). Tako je na primer vprašanje, ali za element
\(g\) grupe \(\Z/_{5}\) velja \(g \in \mathcal{C}^{1}(\mathbb{R})\) formalno smiselno,
vendar tudi povsem v nasprotju z matematično prakso.
V nasprotju s tem so v teoriji tipov elementi nerazdružljivi od svojih tipov; če velja
sodba \(\typejudgment{x}{A}\), potem vprašanja ``Ali velja \(\typejudgment{x}{B}\)?''
sploh ne moremo tvoriti.

Drugič, sodba \(\typejudgment{x}{A}\) ni trditev, ki bi jo lahko dokazali, temveč je
konstrukcija, do katere lahko pridemo po nekem končnem zaporedju \emph{pravil sklepanja}.
To so sintaktična pravila, ki nam omogočijo, da iz določenega nabora
sodb tvorimo nove sodbe. Primer pravila sklepanja bi bilo pravilo, ki pravi, da
lahko iz sodb \(\typejudgment{x}{A}\) in \(\defun{f}{A}{B}\) sklepamo, da velja
sodba \(\typejudgment{f(x)}{B}\), kar predstavlja uporabo funkcije \(f\)
na elementu \(x\).

Poleg sodb oblike \(\typejudgment{x}{A}\) v teoriji tipov obstajajo še tri vrste sodb.
Prvič, obstaja sodba \(\type{A}\), s katero izrazimo, da je tip \(A\) dobro konstruiran.
Primer pravila sklepanja s to sodbo bi bilo
pravilo, ki pravi, da lahko iz sodb \(\type{A}\) in \(\type{B}\) sklepamo, da velja
sodba \(\type{A \to B}\), kar predstavlja konstrukcijo tipa funkcij
med tipoma \(A\) in \(B\).
Drugič, obstajata še sodbi \(\typejudgment{x \doteq y}{A}\) in \(\type{A \doteq B}\),
s katerima izražamo \emph{sodbeno enakost}. Za obe vrsti sodbene enakosti veljajo pravila
sklepanja, preko katerih postaneta ekvivalenčni relaciji,
veljajo pa še določena substitucijska
pravila, ki omogočajo, da lahko v poljubnem izrazu elemente in tipe nadomeščamo s sodbeno
enakimi elementi in tipi. Ker bomo kasneji spoznali še drugo vrsto enakosti, je vredno
poudariti, da je sodbena enakost zelo stroga oblika enakosti. Sodbeno enake izraze lahko
zato razumemo kot različne zapise za \emph{isti} element.

Pomemben element teorije tipov, ki smo ga do sedaj izpuščali, je dejstvo,
da vsaka sodba nastopa v določenem \emph{konstekstu}. Konteksti so končni seznami
deklariranih spremenljivk \(\typejudgment{x_k}{A_{k}}\), ki jih sodbe v tem kontekstu
lahko vsebujejo, razumemo pa jih lahko tudi kot nabor predpostavk, pod katerimi določena
sodba velja.  To, da je \(\mathcal{J}\) sodba v kontekstu \(\Gamma\), zapišemo kot \(\Gamma \vdash \mathcal{J}\).
Primer pravila sklepanja s sodbo v konstekstu bi bilo pravilo, ki pravi,
da lahko iz sodbe
\(\typejudgment{x}{A} \vdash \typejudgment{b}{B}\) sklepamo, da velja
\(\emptyset \vdash \typejudgment{\lambda x.b(x)}{A \to B}\), kar predstavlja konstrukcije
anonimne funkcije z vezavo spremenljivke \(x\).

Razpravo povzamemo v sledeči vzajemno rekurzivni definiciji sodb in kontekstov. To ni popolna definicija Martin-Löfove odvisne teorije tipov, vendar samo nastavek, s pomočjo katerega lahko izrazimo pravila za konstrukcijo različnih tipov in njihovih elementov.

\begin{definicija}
  V Martin-Löfovi odvisni teoriji tipov obstajajo štiri vrste sodb:
\begin{enumerate}
\item \(A\) je tip v kontekstu \(\Gamma\), kar izrazimo z \[\Gamma \vdash \type{A}.\]
\item \(A\) in \(B\) sta \emph{sodbeno enaka tipa} tipa v kontekstu \(\Gamma\),
  kar izrazimo z \[\Gamma \vdash \type{A \doteq B}.\]
\item \(x\) je element tipa \(A\) v kontekstu \(\Gamma\), kar izrazimo z
  \[\Gamma \vdash \typejudgment{x}{A}.\]
\item \(x\) in \(y\) sta \emph{sodbeno enaka elementa} v kontekstu \(\Gamma\), kar izrazimo z
  \[\Gamma \vdash \typejudgment{x \doteq y}{A}.\]
\end{enumerate}
\emph{Kontekst} je induktivno definiran seznam sodb, podan s praviloma:
\begin{enumerate}
\item Obstaja prazen kontekst \(\emptyset\).
\item Če je \(\Gamma\) kontekst in lahko prek obstoječih pravil sklepanja izpeljemo sodbo
  \(\Gamma \vdash \type{A}\), je tedaj \(\Gamma, \typejudgment{x}{A}\) kontekst.
\end{enumerate}
\end{definicija}
Ekspliciten zapis kontekstov in sodb bomo od sedaj deloma izpuščali in o
njih govorili v bolj naravnem jeziku. Ko bomo torej trditve začeli stavki kot je
``Naj bo \(A\) tip in \(x\) element \(A\)...'', smo pravzaprav predpostavili, da velja
sodba \(\type{A}\) in da delamo v kontestku \(\Gamma = \typejudgment{x}{A}\).

S temi definicijami lahko za konec tega podpoglavja predstavimo še specifikaciji
funkcijskih in produktnih tipov. Velik del pravil sklepanja za funkcijske tipe smo že
videli na primerih, tu pa jih še dopolnimo in zberemo v definicijo.

\begin{definicija}
  Naj bosta \(A\) in \(B\) tipa. Tedaj lahko tvorimo \emph{funkcijski tip} \(A \to B\),
  za katerega veljajo sledeča pravila sklepanja:
  \begin{enumerate}
  \item Iz \(\typejudgment{t}{A}\) in \(\defun{f}{A}{B}\) lahko tvorimo
    \(\typejudgment{f(t)}{B}\).
  \item Če lahko pod predpostavko \(\typejudgment{x}{A}\) tvorimo
    \(\typejudgment{b(x)}{B}\), lahko tvorimo tudi \({\typejudgment{\lambda x.b(x)}{A \to B}}\).
  \item Za vsak \(\typejudgment{t}{A}\) velja enakost \((\lambda x.b(x))(t) \doteq b(t)\).
  \item Za vsak \(\defun{f}{A}{B}\) velja enakost \(\lambda x.f(x) \doteq f\).
  \end{enumerate}
  Elemente funkcijskega tipa imenujemo \emph{funkcije}.
\end{definicija}
\begin{definicija}
  Naj bosta \(A\) in \(B\) tipa. Tedaj lahko tvorimo \emph{produktni tip} \(A \times B\),
  za katerega veljajo sledeča pravila sklepanja.
  \begin{enumerate}
  \item Iz \(\typejudgment{t}{A \times B}\) lahko tvorimo \(\typejudgment{\proj{1}(t)}{A}\) in
    \(\typejudgment{\proj{2}(t)}{B}\).
  \item Iz \(\typejudgment{a}{A}\) in \(\typejudgment{b}{B}\) lahko tvorimo
    \(\typejudgment{(a, b)}{A \times B}\).
  \item Za vsaka \(\typejudgment{a}{A}\) in \(\typejudgment{b}{B}\) veljata enakosti
    \(\proj{1}(a, b) \doteq a\) in \(\proj{2}(a, b) \doteq b\).
  \item Za vsak \(\typejudgment{t}{A \times B}\) velja enakost
    \((\proj{1}(t), \proj{2}(t)) \doteq t\).
  \end{enumerate}
  Elemente produktnega tipa imenujemo \emph{pari}.
\end{definicija}

\subsection{Odvisne vsote in odvisni produkti}
Pomemben element \emph{odvisne} teorije tipov je dejstvo, da lahko sodbe \(\type{A}\)
tvorimo v poljubnem kontekstu, to pa pomeni da lahko izrazimo tudi sodbe
oblike \(\typejudgment{x}{A} \vdash \type{B(x)}\).
V tem primeru pravimo, da je \(B\) \emph{odvisen tip},
oziroma, da je \emph{B družina tipov nad A}. Odvisni tipi nam bodo omogočili, da izrazimo
tipe, parametrizirane s spremenljivko nekega drugega tipa.

Za delo z odvisnimi tipi vpeljemo še dve pomembni konstrukciji. Prvič, definiramo
\emph{odvisni produkt tipov A in B}, elementi katerega so predpisi, ki vsakemu elementu
\(\typejudgment{x}{A}\) priredijo element v pripadajočem tipu \(B(x)\).
Če na \(B\) gledamo kot družino tipov
nad \(A\), lahko na elemente odvisnega produkta gledamo kot na \emph{prereze družine B},
pogosto pa jih imenujemo tudi \emph{odvisne funkcije} ali \emph{družine elementov B,
parametrizirane z A}. Drugič, definiramo \emph{odvisno vsoto tipov A in B}, elementi
katere so pari \((x, y)\), kjer je \(\typejudgment{x}{A}\) in \(\typejudgment{y}{B(x)}\).
Z odvisno vsoto tako zberemo celotno družino tipov \(B\) v skupen tip,
na pare \((x, y)\) pa lahko gledemo kot na elemente \(\typejudgment{y}{B(x)}\), označene z
elementom \(\typejudgment{x}{A}\), nad katerim ležijo.
Drugače pogledano lahko na pare \((x, y)\) gledamo tudi kot na elemente
\(\typejudgment{x}{A}\), opremljene z dodatno strukturo \(y\) iz
pripadajočega tipa \(B(x)\).

\begin{definicija}
  Naj bo \(A\) tip in \(B\) družina tipov nad \(A\). Tedaj lahko tvorimo tip
  \[\pitype{x}{A}{B(x)},\] imenovan \emph{odvisni produkt tipov A in B}.
  Zanj veljajo sledeča pravila sklepanja:
  \begin{enumerate}
  \item Iz \(\typejudgment{t}{A}\) in
    \(\typejudgment{f}{\pitype{x}{A}{B(x)}}\) lahko tvorimo
    \(\typejudgment{f(t)}{B(t)}\).
  \item Če lahko pod predpostavko \(\typejudgment{x}{A}\) tvorimo
    \(\typejudgment{b(x)}{B(x)}\), lahko tvorimo tudi
    \(\typejudgment{\lambda x.b(x)}{\pitype{x}{A}{B(x)}}\).
  \item Za vsak \(\typejudgment{t}{A}\) velja enakost \((\lambda x.b(x))(t) \doteq b(t)\).
  \item Za vsak \(\typejudgment{f}{\pitype{x}{A}{B(x)}}\) velja enakost \(\lambda x.f(x) \doteq f\).
  \end{enumerate}
\end{definicija}

\begin{definicija}
  Naj bo \(A\) tip in \(B\) družina tipov nad \(A\). Tedaj lahko tvorimo tip
  \[\sumtype{x}{A}{B(x)},\] imenovan \emph{odvisna vsota tipov A in B}.
  Zanj veljajo sledeča pravila sklepanja
  \begin{enumerate}
  \item Iz \(\typejudgment{t}{\sumtype{x}{A}{B(x)}}\) lahko tvorimo
    \(\typejudgment{\proj{1}(t)}{A}\) in
    \(\typejudgment{\proj{2}(t)}{B(\proj{1}(t))}\).
  \item Iz \(\typejudgment{a}{A}\) in \(\typejudgment{b}{B(a)}\) lahko tvorimo
    \(\typejudgment{(a, b)}{\sumtype{x}{A}{B(x)}}\).
  \item Za vsaka \(\typejudgment{a}{A}\) \(\typejudgment{b}{B(x)}\) veljata enakosti
    \(\proj{1}(a, b) \doteq a\) in \(\proj{2}(a, b) \doteq b\).
  \item Za vsak \(\typejudgment{t}{\sumtype{x}{A}{B(x)}}\) velja enakost
    \((\proj{1}(t), \proj{2}(t)) \doteq t\).
  \end{enumerate}
\end{definicija}

\begin{opomba}
  \label{neodvisna-vsota}
Opazimo lahko podobnost med pravili sklepanja za odvisne produkte in
funkcije ter za odvisne vsote in produkte. Ta podobnost ni naključje, saj če
je \(A\) tip in \(B\) tip, neodvisen od \(A\), lahko nanj kljub temu gledamo kot na tip,
\emph{trivialno} odvisen od \(A\). V tem primeru veljata enakosti
\(\sumtype{x}{A}{B} \doteq A \times B\) in \(\pitype{x}{A}{B} \doteq A \to B\).
\end{opomba}

\begin{primer}
  S pomočjo odvisnih produktov lahko podamo specifikacijo tipa
  \emph{naravnih števil} \(\N\), za katera veljajo sledeča pravila sklepanja.
  \begin{enumerate}
  \item \(\typejudgment{\typename{0}}{\N}\) je naravno število.
  \item Če velja \(\typejudgment{n}{\N}\),
    velja \(\typejudgment{\natsucc(n)}{\N}\).
  \item Velja sledeč \emph{princip indukcije}. Naj bo \(B\) družina tipov nad \(\N\) in
    denimo, da veljata \(\typejudgment{b_{\natzero}}{B(\natzero)}\) in
    \(\typejudgment{b_{\natsucc}}{\pitype{n}{\N}{(B(n) \to B(\natsucc(n)))}}\).

    Tedaj velja
    \(\typejudgment{\typename{ind}_{\N}(b_{\natzero}, b_{\natsucc})}{\pitype{n}{\N}{B(n)}}\).
  \item Veljata enakosti
    \label{naturals}
    \(\typename{ind}(b_{\natzero}, b_{\natsucc}, \natzero) \doteq b_{\natzero}\) in
    \(\typename{ind}(b_{\natzero}, b_{\natsucc}, \natsucc(n)) \doteq
      b_{\natsucc}(n, \typename{ind}(b_{\natzero}, b_{\natsucc}, n)\).
  \end{enumerate}
  Princip indukcije nam pove, da če želimo za vsako naravno število \(\typejudgment{n}{\N}\)
  konstruirati element tipa \(B(n)\), tedaj zadošča, da konstruiramo element tipa \(B(\natzero)\)
  in pa da znamo za vsak \(\typejudgment{n}{\N}\) iz elementov tipa \(B(n)\) konstruirati elemente
  tipa \(B(\natsucc(n))\), v čemer lahko prepoznamo podobnost z običajnim principom indukcije v
  teoriji množic.
\end{primer}

\subsection{Interpretacija \emph{izjave kot tipi}}

S tipi ne želimo predstaviti samo matematičnih objektov, temveč tudi matematične trditve,
kar bomo dosegli preko tako imenovane interpretacije \emph{izjave kot tipi}
(ang. \emph{propositions as types}). V tej interpretaciji vsaki trditvi \(P\) dodelimo
tip z istim imenom, elementi katerega so dokazi oz.~\emph{priče}, da ta trditev velja.
Konstrukcija
elementa \(\typejudgment{p}{P}\) tako postane konstrukcija priče \(p\), ki potrjuje
veljavnost trditve \(P\), to pa ustreza dokazu trditve \(P\).

Če si sedaj iz vidika izjav kot tipov zopet ogledamo pravila sklepanja za produktne in
funkcijske tipe, lahko v njih prepoznamo
pravila sklepanja za konjunkcijo in implikacijo. Dokaz trditve \(P \wedge Q\) je namreč sestavljen
iz para dokazov posameznih trditev \(P\) in \(Q\), dokaz implikacije \(P \Rightarrow Q\) pa lahko razumemo
kot predpis, ki iz dokazov \(P\) konstruira dokaze \(Q\). Uporaba funkcije \(P \to Q\)
tako ustreza pravilu \emph{modus ponens}, ki pravi, da lahko iz veljavnosti izjav \(P\) in \(P \Rightarrow Q\)
sklepamo veljavnost izjave \(Q\).

Denimo sedaj, da je \(A\) tip in \(P\) družina tipov nad \(A\), vsak od katerih predstavlja določeno
izjavo. Tako družino izjav imenujemo tudi \emph{predikat na A}.
Tedaj lahko logično interpretacijo podamo tudi odvisni vsoti in odvisnemu produktu;
ustrezata namreč eksistenčni in univerzalni kvantifikaciji. Elementi odvisnega vsote
\(\sumtype{x}{A}{P(x)}\) so pari \((x, p)\), kjer je \(p\) dokaz \(P(x)\), zato lahko nanje gledamo
kot na priče veljavnosti \(\exists \typejudgment{x}{A}. P(x)\), elementi odvisnega produkta
\(\pitype{x}{A}{P(x)}\) pa so predpisi, ki vsakemu elementu \(\typejudgment{x}{A}\) priredijo
dokaz izjave \(P(x)\), zato lahko nanje gledamo kot na priče veljavnosti
\(\forall \typejudgment{x}{A}.P(x)\).

Definicije predikatov in formulacije trditev, ki jih bomo predstavili v preostanku dela, bomo
izrazili v jeziku interpretacije \emph{izjave kot tipi}, kljub temu pa se lahko eksplicitnemu
zapisu prič pogosto izognemo. V dokazih trditev bomo tako pogosto uporabljali naravni
jezik sklepanja, vedno pa pravzaprav konstruiramo element določenega tipa, ki predstavlja trditev, ki jo dokazujemo.

\subsection{Tip identifikacij}
Osnovna motivacija za vpeljavo tipa identifikacij je dejstvo, da je pojem sodbene enakosti
\emph{prestrog} in da prek nje pogosto ne moremo identificirati vseh izrazov, ki bi
jih želeli. Poleg tega si v skladu z interpretacijo \emph{izjave kot tipi} želimo tip, elementi
katerega bi predstavljali dokaze enakosti med dvema elementoma.

Pomankljivost sodbene enakosti si lahko ogledamo na primeru
komutativnosti naravnih števil.

\begin{primer}
  Seštevanje naravnih števil bomo definirali kot funkcijo
  \[\defun{\typename{sum}}{\N}{(\N \to \N)},\] njeno evaluacijo \(\typename{sum}(m, n)\)
  pa bomo kot običajno pisali kot \(m + n\).
  Ker je tip \(\N \to (\N \to \N)\) po opombi \ref{neodvisna-vsota} sodbeno enak tipu
  \(\pitype{n}{\N}{\N \to \N}\), lahko njegove elemente konstruiramo po indukcijskem
  principu naravnih števil.
  Izpuščajoč nekaj podrobnosti to pomeni, da želimo vrednost izraza \(m + n\) definirati z
  indukcijo na \(n\), torej moramo v baznem primeru podati vrednost izraza \(m + \natzero\),
  v primeru koraka pa moramo pod predpostavko, da poznamo vrednost izraza \(m + n\),
  podati vrednost izraza \(m + \natsucc(n)\). Naravno je,
  da izberemo \(m + \natzero :\doteq m\) in \(m + \natsucc(n) :\doteq\natsucc(m + n)\).
  Za poljubni \emph{fiksni} naravni števili, recimo
  \(\typename{4}\) in \(\typename{5}\), lahko sedaj z uporabo pravil sklepanja za naravna
  števila pod točko \ref{naturals} dokažemo, da velja
  \(\typename{4} + \typename{5} \doteq \typename{5} + \typename{4}\). Če pa sta po drugi strani
  \(m\) in \(n\)
  \emph{spremenljivki} tipa \(\N\), tedaj zaradi asimetričnosti definicije seštevanja
  enakosti \(n + m \doteq m + n\) ne moremo
  dokazati. Potrebujemo tip \emph{identifikacij} \(\Id{\N}\), z uporabo katerega bi
  lahko z indukcijo pokazali, da velja \[\pitype{m, n}{\N}{\Id{\N}(n + m, m + n)}.\]
\end{primer}

\begin{definicija}
  Naj bo \(A\) tip in \(\typejudgment{a, x}{A}\). Tedaj lahko tvorimo tip
  \(\Id{A}(a, x)\),
  imenovan \emph{tip identifikacij med elementoma a in x}. Zanj veljajo sledeča pravila
  sklepanja:
  \begin{enumerate}
  \item Za vsak element \(\typejudgment{a}{A}\) obstaja identifikacija
    \(\typejudgment{\refl{a}}{\Id{A}(a, a)}\).
  \item Velja sledeč princip \emph{eliminacije identifikacij}.
    Naj bo \(\typejudgment{a}{A}\) in \(B\) družina tipov, indeksirana z
    \(\typejudgment{x}{A}\) ter \(\typejudgment{p}{\Id{A}(a, x)}\). Denimo, da velja
    \(\typejudgment{r(a)}{B(a, \refl{a})}\). Tedaj velja
    \[\typejudgment{\typename{ind}_{\Id{A}}(a, r(a))}
      {\pitype{x}{A}{\pitype{p}{\Id{A}(a, x)}{B(x, p)}}}\]
  \item Za vsak \(\typejudgment{a}{A}\) velja enakost
    \(\typename{ind}_{\Id{A}}(a, r(a))(a, \refl{a}) \doteq r(a)\).
  \end{enumerate}
  Tip identifikacij \(\Id{A}(a, x)\) bomo od sedaj pisali kar kot \(a = x\) in za boljšo
  berljivost izpuščali ekspliciten zapis tipa \(A\). Kadar je \(p\) identifikacija
  med \(a\) in \(x\), bomo torej pisali \(\eqtype{p}{a}{x}\).

  Konstruktor \(\eqtype{\refl{a}}{a}{a}\)
  nam zagotavlja, da za vsak element \(\typejudgment{a}{A}\) obstaja kanonična
  identifikacija elementa samega s sabo. Princip eliminacije identifikacij nato zatrdi,
  da če želimo za vsako identifikacijo \(\eqtype{p}{a}{x}\) konstruirati element tipa
  \(B(x, p)\), tedaj zadošča, da konstruiramo le element tipa \(B(a, \refl{a})\).
  Pravimo tudi, da je družina tipov \(a = x\) \emph{induktivno definirana} s
  konstruktorjem \(\refl{a}\).
\end{definicija}

Preko principa eliminacije identifikacij lahko dokažemo mnoge lastnosti, ki bi jih od enakosti pričakovali. Nekaj od teh vključuje:

\begin{enumerate}
\item Velja \emph{simetričnost}: za vsaka elementa \(\typejudgment{x, y}{A}\) obstaja
  funkcija \[\defun{\typename{sym}}{(x = y)}{(y = x)}.\] Identifikacijo \(\typename{sym}(p)\)
  imenujemo \emph{inverz} identifikacije \(p\) in ga pogosto označimo z \(p^{-1}\).
\item Velja \emph{tranzitivnost}: za vsake elemente \(\typejudgment{x, y, z}{A}\) obstaja
  funkcija \[\defun{\typename{concat}}{(x = y)}{((y = z) \to (x = z))},\] ki jo imenujemo
  \emph{konkatenacija identifikacij}, \(\typename{concat}(p, q)\) pa pogosto označimo
  z \(p \cdot q\).
\item Funkcije ohranjajo identifikacije: za vsaka elementa \(\typejudgment{x, y}{A}\)
  in vsako funkcijo \(\defun{f}{A}{B}\) obstaja funkcija
  \[\defun{\typename{ap}_{f}}{(x = y)}{(f(x) = f(y))},\]
  ki jo imenujemo \emph{aplikacija} funkcije \(f\) na identifikacije v \(A\).
\item Družine tipov spoštujejo identifikacije: za vsaka elementa
  \(\typejudgment{x, y}{A}\) in družino tipov
  \(B\) nad \(A\) obstaja funkcija
  \[\defun{\typename{tr}_{B}}{(x = y) \to B(x)}{B(y)},\]
  imenovana \emph{transport vzdolž identifikacij}.
  Ker sta elementa \(x\) in \(y\) identificirana, lahko preko transporta dokažemo, da sta
  tudi tipa \(B(x)\) in \(B(y)\) v bijektivni korespondenci. Pokažemo lahko namreč, da
  velja \(\transp{B}(p) \circ \transp{B}(p^{-1}) = \id{}\) in obratno.

  V interpretaciji \emph{izjave kot tipi} nam transport pove tudi, da lahko znotraj
  predikatov substituiramo identificirane elemente. Če je namreč \(P\) predikat
  na \(A\) in velja identifikacija \(x = y\), tedaj \(P(x)\) velja natanko tedaj kot
  \(P(y)\).

\end{enumerate}
Konstrukcije naštetih funkcij lahko poiščemo v 5.~poglavju (TODO cite intro to hott), vse
izmed njih pa zahtevajo preprosto, a nekoliko tehnično, uporabo eliminacije identifikacij.

\pagebreak

V tem delu bomo pogosto imeli opravka z identifikacijami znotraj odvisnih vsot, torej z identifikacijami oblike \((x, y) = (x', y')\), kjer sta
\(\typejudgment{(x, y), (x', y')}{\sumtype{x}{A}{B(x)}}\). Izkaže se, da imajo identifikacije te oblike preprostejšo karakterizacijo, ki se je bomo pogosto poslužili.
Velja namreč naslednje:
\begin{trditev}
  Naj bosta \(\typejudgment{(x, y), (x', y')}{\sumtype{x}{A}{B(x)}}\). Identifikacije med \((x, y)\) in \((x', y')\) lahko tedaj karakteriziramo kot par identifikacij
  \(\eqtype{p}{x}{x'}\) in \(\transp{B}(p, y) = y'\).
\end{trditev}
Formalna formulacija in uporaba te trditve je zunaj obsega tega dela. Namesto tega bomo identifikacije znotraj odvisnih vsot vedno preprosto nadomestili z pari identifikacij, kot opisano. Trditve tudi ne bomo dokazali, povemo pa lahko nekaj o njeni interpretaciji. TODO

Vredno je omeniti sledeča posebna primera:
\begin{enumerate}
\item Funkcija transporta vzdolž identifikacije \(\refl{a}\) je sodbeno enaka identiteti na \(B(a)\) za vsak \(\typejudgment{a}{A}\). Če na prvi komponenti identifikacije znotraj odvisne vsote torej vzamemo \(\refl{a}\), nam na drugi komponenti ostanejo le identifikacije med \(y\) in \(y'\) v \(B(a)\).
\item Če je tip \(B\) neodvisen od \(A\), je funkcija transporta
  \(\defun{\transp{B}(p)}{B}{B}\) enaka identiteti za vsako identifikacijo \(p\) v \(A\).
  Ker je odvisna vsota \(\sumtype{x}{A}{B}\) enaka produktu \(A \times B\), nam karakterizacija pove, da so identifikacije znotraj tipa \(A \times B\) sestavljene iz dveh neodvisnih identifikacij znotraj tipov \(A\) in \(B\).
\end{enumerate}
%%% Local Variables:
%%% mode: latex
%%% TeX-master: "../Najdovski-27191110-2023"
%%% End:
