\section{Osnovne definicije homotopske teorije tipov}

\subsection{Odvisne vsote in odvisni produkti}

\begin{definicija}
  Naj bo \(A\) tip in \(B\) družina tipov nad \(A\).
  Definiramo tip \[\sumtype{x}{A}{Bx},\]
  imenovan \emph{odvisna vsota tipov A in B}. Elementi \(\sumtype{x}{A}{Bx}\) so pari
  \(\left(x, y\right)\), kjer je \(\typejudgment{x}{A}\) in \(\typejudgment{y}{Bx}\).
\end{definicija}

\begin{definicija}
  Naj bo \(A\) tip in \(B\) družina tipov nad \(A\).
  Definiramo tip \[\pitype{x}{A}{Bx},\]
  imenovan \emph{odvisni produkt tipov A in B}. Elementi
  \(\pitype{x}{A}{Bx}\) so predpisi \(f\), ki vsakemu elementu \(\typejudgment{x}{A}\)
  priredijo element \(\typejudgment{f(x)}{Bx}\), imenujemo pa jih
  \emph{odvisne funkcije med A in B}.
\end{definicija}


\subsection{Enakost in homotopija}

\begin{definicija}
  Naj bo \(A\) tip in \(\typejudgment{x, y}{A}\). Definiramo tip \(x = y\), imenovan
  \emph{tip identifikacij med x in y}, elemente katerega pa imenujemo
  \emph{identifikacije med x in y}. Za vsak \(\typejudgment{x}{A}\) obstaja identifikacija
  \(\eqtype{\refl{x}}{x}{x}\). TODO induction principle
\end{definicija}

Iz indukcijskega pravila za tip identifikacij lahko izpeljemo mnoge znane lastnosti
enakosti. Pokazali bomo, da obstoj identifikacij tvori ekvivalenčno relacijo in da vsaka
funkcija to relacijo ohranja,
konstruirali pa bomo še tako imenovano funkcijo \emph{transport}.

\begin{definicija}
  Naj bo \(A\) tip. Definiramo operacijo \emph{konkatenacije}
  \[\typejudgment{\typename{concat}}{\pitype{x, y, z}{A}
      {(x = y) \to ((y = z) \to (x = z))}}\]
  in operacijo \emph{inverza}
  \[\typejudgment{\typename{inv}}{\pitype{x, y}{A}{(x = y) \to (y = x)}}.\]
  Za \(\typename{concat}(x, y, z, p, q)\) bomo pogosteje pisali \(p \cdot q\), za
  \(\typename{inv}(x, y, p)\) pa \(p^{-1}\).
\end{definicija}

\begin{konstrukcija}
  Da bi definirali operacijo \(\typename{concat}\), po indukcijskem pravilu za
  identifikacije zadošča, da podamo \(\typename{concat}(x, x, z, \refl{x}, q)\)
\end{konstrukcija}

\begin{trditev}
  \label{eq-Sigma}
  TODO karakterizacija enakosti v odvisnih vsotah
\end{trditev}

\begin{definicija}
  Naj bosta \(\defun{f, g}{A}{B}\) funkcij. Pravimo, da sta funkciji \(f\) in \(g\)
  \emph{homotopni}, če obstaja element tipa
  \[f \sim g := \sumtype{x}{A}{f(x) = g(x)}.\]
  Elemente \(\homotopy{H}{f}{g}\) imenujemo \emph{homotopije med f in g}.
\end{definicija}

%%% Local Variables:
%%% mode: latex
%%% TeX-master: "../Najdovski-27191110-2023"
%%% End:
