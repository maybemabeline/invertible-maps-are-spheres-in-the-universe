\section{Osnovne definicije homotopske teorije tipov}

\subsection{Odvisne vsote in odvisni produkti}

\begin{definicija}
  Naj bo \(A\) tip in \(B\) družina tipov nad \(A\).
  Definiramo tip \[\sumtype{x}{A}{Bx},\]
  imenovan \emph{odvisna vsota tipov A in B}. Elementi \(\sumtype{x}{A}{Bx}\) so pari
  \(\left(x, y\right)\), kjer je \(\typejudgment{x}{A}\) in \(\typejudgment{y}{Bx}\).
\end{definicija}

\begin{trditev}
  \label{eq-Sigma}
  TODO karakterizacija enakosti v odvisnih vsotah
\end{trditev}

\begin{definicija}
  Naj bo \(A\) tip in \(B\) družina tipov nad \(A\).
  Definiramo tip \[\pitype{x}{A}{Bx},\]
  imenovan \emph{odvisni produkt tipov A in B}. Elementi
  \(\pitype{x}{A}{Bx}\) so predpisi \(f\), ki vsakemu elementu \(\typejudgment{x}{A}\)
  priredijo element \(\typejudgment{f(x)}{Bx}\), imenujemo pa jih
  \emph{odvisne funkcije med A in B}.
\end{definicija}

\subsection{Enakost in homotopija}

\begin{definicija}
  Naj bo \(A\) tip in \(\typejudgment{x, y}{A}\). Definiramo tip \(x = y\), imenovan
  \emph{tip identifikacij med x in y}, elemente katerega imenujemo
  \emph{identifikacije med x in y}. Za vsak \(\typejudgment{x}{A}\) obstaja identifikacija
  \(\eqtype{\refl{x}}{x}{x}\).
\end{definicija}

%%% Local Variables:
%%% mode: latex
%%% TeX-master: "../Najdovski-27191110-2023"
%%% End:
