\section{Uvod}

Teorije tipov so logični formalizmi, ki slonijo na objektih, imenovanih tipi, opremljenih s tako imenovanimi konstruktorji indukcijskimi principi. Z razliko od teorije množic ne predpostavimo obstoja vesolja množic, znotraj katerega prepoznamo in kodiramo matematične objekte, kot so produkti, funkcije in naravna števila, vendar le te aksiomatiziramo kot prvorazredne objekte. Konstruktorji nam povejo, kako lahko tvorimo njihove elemente, indukcijski principi pa nam povejo, kako o njih dokazujemo trditve in na njih gradimo konstrukcije.

Osnove sodobne teorije tipov segajo v leto 1971, ko je švedski matematik Per Martin-Löf v neobjavljenem članku prvič opisal sistem, obliko katerega dandanes imenujemo Martin-Löfova odvisna teorija tipov. Dve leti pozneje je teorijo razširil in takrat tudi prvič podal induktivno definicijo tipa identifikacij, elementi katerega predstavlajo dokaze enakosti med elementi drugih tipov \cite{Martin-Lof-1973}. Tip identifikacij v tej obliki -- imenovani intezionalna enakost -- dolgo časa ni bil dobro razumljen, njegova raba je začela zamirati in pojavljati so se začele tudi druge teorije tipov z drugačnimi pojmi enakosti, imenovane ekstenzionalna enakost.

Ponovno zanimanje za intenzionalno enakost je začelo vznikati v prvem desetletju tega tisočletja, ko se je za teorijo tipov začel zanimati zdaj pokojni Vladimir Vojevodski. Ta je izhajal iz področja homotopske teorije, na katerem je pred vstopom v teorijo tipov prejel Fieldsovo priznanje. Za petami pa je imel zaporedje dogodkov, v katerem so se dokazi večih matematikov na področju homotopske teorije, vključno z njim, vrsto let po objavi izkazali za napačne. To je v njem vzbudilo zanimanje za teorije tipov, saj na njih slonijo dokazovalni pomočniki; programski jeziki, znotraj katerih je moč izraziti in dokazati matematične trditve \cite{origin-of-uni-foundations}.

Izkušnje Vojevodskega na področju homotopske teorije pa so v teorijo tipov pripeljale novo interpretacijo. Tipe lahko namreč interpretiramo kot topološke prostore, elemente intenzionalnega tipa identifikacij pa kot poti v teh prostorih. S tem je dodatno interpretacijo dobila tudi tako imenovana \emph{relevanca dokazov} (ang. proof relevance); ideja iz konstruktivne matematike, ki pravi, da dokazov, predvsem dokazov enakosti, ne smemo zavreči, temveč da lahko nosijo pomembne informacije. Ločeni dokazi enakosti se med seboj lahko razlikujejo, analogno pa med dvema točkama v prostoru lahko konstruiramo več različnih poti. Kljub temu se dva dokaza lahko izkažeta za različna na nebistven način, to pa je analogno potema v prostoru, med katerima lahko konstruiramo homotopijo.

Vojevodski je v teorijo tipov vpeljal vrsto pomembnih konceptov, vsak izmed katerih je izvirno izhajal iz homotopske teorije \cite{origin-of-uni-foundations}. Prepoznal je, da je struktura identifikacij v tipih za njihovo razumevanje ključna in vpeljal koncepte kot so kontraktibilnost, propozicije in množice, vse izmed katerih bomo v tem delu spoznali. Ta pristop do teorije tipov pogosto imenujemo tudi homotopska teorija tipov.

Prepoznal pa je tudi, da je dosedanje razumevanje ekvivalence med tipi v določenem smislu zmotno, zmotno pa je natanko v primeru, ko tipi pod vprašanjem nosijo višjo homotopsko strukturo. Prejšnji pojem ekvivalence bomo v tem delu imenovali obrnljivost, popravljeni pojem Vojevodskega pa ekvivalenca. Želimo namreč, da je pojem ekvivalence \emph{lastnost} funkcij, pojem obrnljivosti pa je na funkcijah predstavljal \emph{strukturo}. To je v konstrukcije z obrnljivostjo vpeljalo nepotrebne arbitrarne izbire, najpomembneje pa je pojem obrnljivostji nekompatibilen z Vojevodskijevem aksiomom univalence. Ta aksiom zatrdi, da se dokazi enakost med tipi sklada z ekvivalencami med njimi; ideja, ki je za razvoj homotopske teorije v teoriji tipov ključna. Če pa bi analogno postavili aksiom, ki zatrdi, da se dokazi enakost med tipi skladajo z obrnljivimi funkcijami med njimi, bi teorija postala dokazljivo nekonsistenta.

S tem se je zgodba obrnljivosti do neke mere končala; Vojevodski je kvalitativno prepoznal, da je s pojmom v določenih primerih nekaj narobe in ga nadomestil. V tem delu pa se k obrnljivosti ponovno obrnemo in podamo kvantitaivno obravnavo, ki karakterizira povezavo med obrnljivostjo in ekvivalenco. S tem postane jasneje, kako višja homotopska struktura v tipih vpliva na njuno razliko. Ne nazadnje karakterizacijo uporabimo za to, da poiščemo nepričakovano povezavo med obrnljivostjo in homotopskim tipom sfere.

V prvem poglavju kratko predstavimo osnove Martin-Löfove odvisne teorije tipov, njena pravila sklepanja in intezionalni tip identifikacij. V drugem poglavju definiramo pojma obrnljivosti in ekvivalence ter kvalitativno predstavimo razliko med njima, spotoma pa vpeljemo več pomembnih konceptov in elementarnih izrekov homotopske teorije tipov. V zadnjem poglavju dokažemo nekaj močnejših izrekov homotopske teorije tipov in nazadnje predstavimo našo karakterizacijo.

Rezultata zadnjega poglavja sta formalizirana v dokazovalnem pomočniku Agda, ki implementira Martin-Löfovo teorijo tipov. To pomeni, da je bila pravilnost dokazov v celoti računalniško preverjena in sloni le na konsistenci teorije tipov in njeni pravilni implementaciji v Agdi. Formalizacija je bila narejena s pomočjo knjižnice 1Lab, dostopne na povezavi \cite{1lab}, sama formalizacija pa je dostopna na github repozitoriju dela \cite{repo}.

%%% Local Variables:
%%% mode: LaTeX
%%% TeX-master: "../Najdovski-27191110-2023"
%%% End:
